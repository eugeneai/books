\documentclass{intech}
\usepackage[nolist]{acronym}
\usepackage{graphicx}
\usepackage{hyperref} % FIXME setup the package.
\usepackage{hypernat}
%\usepackage{subfig}
\usepackage{color}
\usepackage{multirow,multicol}
\usepackage{booktabs}
\usepackage{rotating}

%\usepackage{listings,program,threeparttable}
%\usepackage{algorithmicx,algpseudocode}
\usepackage{minted}

\setcounter{chapter}{11} % List of chapters http://www.intechopen.com/booksprocess/allchapters/chapter/112248
\booktitle{Will-be-set-by-IN-TECH}%

\chaptertitle{Information Systems Framework Synthesis\\ on the Base of a Logical Inference}

\authors{Evgeny Cherkashin, Vyacheslav Paramonov,\\ Roman Fedorov, Igor Bychkov}
\affiliation{Matrosov Institute for System Dynamics and Control Theory\\ of Siberian Branch of Russian Academy of Sciences, \\
Irkutsk National Research Technical University, Irkutsk}
\country{Russia}

\begin{document}

\maketitle

\section{Introduction}
Any software development life cycle consists of distinct stages and involves various formal and informal models, agents (designers, software developers, users, etc.) and technologies. Various combinations of stages form a number of software development life cycle schemes, such as waterfall and spiral models, V-model, agile and extreme approaches, iterative and incremental development, and various improvement models [1]. All the approaches make use of models of various degrees of abstraction and formalization. We consider a general case of the life cycles as a process of adaptation of new ideas, requirements and specifications, where each adaptation event is a model change. The adaptation implies modification of all the affected models due to the event. Thus, the software development process is represented as propagation of the differences (modifications) within a set of models representing the software under development. The target of the research is to construct an approach to describe the process of the difference propagation as a basis of a corresponding instrumental environment for software development.

Since 2001 OMG exploits Model Driven Architecture (MDA) of software development. MDA [2] is a part of the model considered in the section III. MDA exploits three levels of abstractions to represent software: CIM, PIM and PSM. The Computation Independent Model (CIM) reflects software’s external requirements – its interfaces. CIM hides structural elements, and can be used for define specifications and checking requirements.

The software designing technique of MDA is based on multistage transformation of Platform Independent Model (PIM) into a number of Platform Specific Models (PSM). PIM is a model of the software reflecting most of the structural and some semantic aspects of the software, but the model contains no information about implementation of the structures on the target program architecture. UML Class Diagram extended with some tag values and additional stereotypes is an example of PIM [3]. The extension allows one to denote implementation variants and hints for structures. PSM is a model, which canimplemented as source code of the subsystems, e.g.,could be a physical structure of a rational database, which is directly (deductively or by means of code templates) translated into DDL SQL-requests.

The transformation of the PIM into PSMs is carried out under control of a Platform Model (PM) and a transformation scenario. PM contains information and algorithms of PIM’s structure analysis and generation of corresponding structures in PSMs. Sometimes PSM is understood as specified variant of PIM. The tag values and stereotypes are used to direct the transformation of a structure into desired frame.

Main advantages of MDA usage in the software development are as follows:
\begin{itemize}
\item Design stage independence of the implementation platform; capability to replace the platform without redesigning PIM.

\item Formal definition of PM: programmers’ knowledge is
represented as rules and algorithms.

\item Raising the automation level of the life cycle: early stage modifications (design stages) are less expensive to implement in PSMs. MDA is a great approach and successfully used in development complex software, but it has significant disadvantage, which we are to overcome:

\item Using the MDA in simple projects usually extends time of software construction, although obtained formal PIM and PM models when analyzed could be used in other projects;

\item Currently MDA is of little use in already constructed and implemented systems and systems based on stored data manipulation, e.g., existing informational systems, as modification of information data model results in database structure modification like adaptation to new data structures; Modification of PIM and source code is ignored by the procedures of transformations.
\end{itemize}

(?) One of the approaches for implementing concept of generative  programming, is MDA. MDA is based on a set of models to be modified in oreder to obtain the source code. (?) However, the specification of the software to be gnerated does not prescribe the ways of transformation of the models. Consider the model formalisms of information systems (IS), their representation and transformation.

(!) According to MDA the process of creating the IS begins with the engineering a so called Computing independent model (CIM)) or Platform Independent Model (PIM) \cite{b115}. MDA does not prescribe the construction of CIM in every project and it is optional. CIM describes general requirements to the system, dictionary of the terms used, and operating conditions (the environment). The model must not contain any information of technical nature, descriptions of data structures and functionality of the system. (Here we add Juring addition)  CIM is to be as general as possible and not dependent on the implementation of the systems. This model is implemented frequently in Unified Modeling Language (UML) [l46], and further enriched and refined.

CIM refinements lead to specification og PIM. We have no experience of usage CIM in IS specifications, but \ldots.

PIM is the basis of MDA. Using PIM IS is described on an abstract level, which allows us to consider the IS under development as independent of the specific properties of software and hardware platform, where it is to be implementaed. According to MDA, PIM is recommended to contain a high degree of detail, and use a high-level platform independent programming language for describing IS functions. However, in the MDA specifications, the details of describing the behavior of the system from the user point of view must be free of thee details of its implementation: the latter should not be present in the PIM. One way to represent PIM is UML, and in a development process UML diagrams are the language for formal description of the system, as well as means of documenting. A very specific implementation aspects of subsystems are described with custom UML tags values and stereotypes.

The source PIM in a general case contains formalized constraints specified in the Object Constraint Language (OCL) [70, 156]. The constraints play an important role in the design and implementation of the IS, they define the properties of the domain and its implementation perculiarities. (about compromize OCL representation of formatm mat theory of sets and informal natural languages....for constraint definition.) Usage of a natural language for constraint definition is not practical as its translators are complex and language terms are not uniquely interpreted. The sets theory and mathematical language are the means of construction of the precise uniquely interpreted formalisms for a constraint definition language, but it is not convenient for a regular programmer. (SKIP) OCL is a compramise formalism that aloso supports the object-oriented traditions. (SKIP) (SKIP)

(SKIP) (?) Consistency criteria and preperties of model structure elements are defined with OCL, such as. (SKIP) (SKIP) 

invariant constraints, which must hold after object instance creation till its destruction;

(?) guarding conditions; (SKIP) guarging conditions are written in brackets and interpreted as triggers; 

definition of pre- and postconditions of operations; (SKIP)

 The OCL constraint is defined within a context denoted as a name of instance or class. Execution result of the OCL equation is a boolean value. (SKIP) OCL includes three operations:

atribute value evaluation;

association transition;

sending message to an instance and receiving the answer.

(?) OCL allows one to track the class instances, and the instances of descendant classes.

After PIM are sufficiently detailed, the Platform specific model (PSM) is constructed \cite{b115}. PSM is a framework model of the IS, where the IS functionality is represented taking into account the properties of the target software and hardware implementation platform.

The transformation from PIM to PSM is carried out according to a formal representation of the properties of the target platform, i.e., Platform Model (PM). PSM is constructed by means of adding details about target platform into PIM.  Можно сказать, что PSM является описанием ИС, представляющим ее на уровне программного кода, т. е. со значительной детализацией особенностей реализации по сравнению с PIM. С учетом всех требований к программно"=аппаратной платформе на основе PSM синтезируется программный код ИС. A general scheme of PIM to PSM transformation is presented in Figure 6.1. 6.1

Figure. Scheme of IS framework development according to MDA


ABOUT CIM (\cite{jurin} et al)

At present (2005...??) PMs in most of commercial MDA systems have been implemented on the basis of algorithmic approach. They are not far from CASE systems translating UML diagrams into a source code by various plug-ins. The main idea of MDA is to allow developer to modify PM according his/her preferences and task properties. Our experience shows that usage of present logical languages and PMs based on formalized knowledge [3] allows us to affect the transformation in an efficient way by means of changing a rule set content. Moreover, declarative paradigm of the logical languages naturally forces programmer to create rules processing only small parts of PIM in a multistage fashion with a number of intermediate decisions. This results in dividing the transformation process on stages, where each next stage deals with more concrete structures, which are nearer to PSM. Such approach looks naturally similar to above described application of the theory.

problematic of local firm traditions and open source.

\section{MDA today}
\label{sec:mda-today}

General classification XMLTRANSFORM, graph conversion (Paramonov disser) SQL-like pattern recodgniniton (kuzmin)

ATL

Stratego



\section{Architecture implication}
\label{sec:arch-imp}

Enterprise information systems (IS) roughly can be divided to three main subsystems: data warehouse, business logics (also referred as business intelligence and sometimes also middleware) and user interface. The design and implementation techniques of the IS are almost always based on this kind division.

Data warehouse is realized as à rational database with a query language, е.g., SQL. Business logics is responsible to support object-relational layer and define object interaction at runtime. User interface is used to modify data in the data warehouse, control the process of the IS functioning and initiate processes of solution of specific tasks such as data-analysis procedures and report generation. The division is principally compatible with standard MVC-scheme (Model-View-Controller). Data warehouse is à Model. User interface is à View. Business logics is à Controller.

The design approaches to realize an the IS can be divided to two categories. The first category is a construction of the IS on the base of some universal library of modules. This approach embodies famous enterprise IS development platform SAP R/3 [1]. The second category is à construction of the IS from the ground using à CASE-system, е.g., Rational Rose. The RAD-systems (Rapid Application Development) supposed to belong to the first category.

First approach can be seen as à process of experience accumulation in various IS manufacturing and consequent implementation of library modules with à reasonable level of abstraction. Each module implements à class of operations or business processes, and can be adjusted to the peculiarities of the business processes of à concrete enterprise. In this case the IS is à composition of previously specified modules implementing à system of business processes ойле enterprise.

Second approach is to use some abstract modeling language to express à design of the system on à humanunderstandable level, and generate the source code of the IS according to the design. The UML language \cite{b2} is widely accepted standard of visual model formal
representation.

Most of present CASE tools contain the subsystems of code generation. The UML language is à general design description language and has no special features oriented on enterprise IS description, е.g., some many"=to-one associations are relations of an instance attribute to its list of values from à reference book. In this case à special choosing widget can be used in à user interface to fill in the attribute with value from the list. Code generators incorporated in the CASE"=tools will generate source code of SQL database definition script and Java classes. The database and Java classes will not interact without modules implementing business logics, coupling the database and classes' instances. User interface usually generated as à result of à database definition script analysis.

Another fundamental problem is the database structure modification in IS administration, which results in necessity to find and modify source code of business logics and user interface. It seems, that the problem cannot be solved with present CASE"=tools.

In late 2001 OMG (Object Management Group) proposed MDA (Model Driven Architecture) \cite{b3}. Software development in the MDA starts with à construction of Platform"=Independent Model (PIM) of an application's business functionality and behavior, typically built in the 11М1 language. This model remains stable as technology evolves. MDA development tools, available now from many vendors, convert the PIM first to à Platform"=Specific Model (PSM) and then to à working implementation on virtually any middleware platform: Web Services, XML/SOAP, EJB, С№/.Net, etc. Portability and interoperability are built into the architecture. OMG's industry"=standard modeling specifications support the MDA: UML; the MetaObject Facility (MOF); the Common Warehouse Metamodel (CWM); and ХМЬ Metadata Interchange (XMI). OMG Task Forces organized around industries including Finance, Manufacturing, Biotechnology, Space technology, and others use the MDA to standardize facilities in their domains.

The ann of our research is to construct à generator software, which can generate à kernel of the IS on the base of an MDA.~The generation is carried out by means sequential specification of the PIM, resulting PSM and, then, the source code generation. To approach the aim an enrichment of UML is considered. The enrichment does not require to change the UML standards. MDA"=model of the IS of consists of UML Class diagrams enriched by stereotypes and tag values, PIM"=PSM conversion knowledge base formalized as Prolog language rules, and constraints on attribute values and associations represented with the OCL"=language. Other kinds of diagrams are not used on the present stage of this research.

Hereafter we will use the following point of view to the structure of the IS.~The IS is constructed of the following program aspects: data warehousing, business"=logics and user interface. The aim is to generate all the program aspects from one PIM model simultaneously so that there were no functional contradictions among their source codes.

\section{Model of IS}
\label{sec:model-is}

The proposed PIM of the IS has two parts: model MS of component structures and the associations and model МС of constraints. The PIM"=PSM conversion is defined by model MI of semantic interpretation of the PIM components and associations in PSM.

\subsection{Model of component structures and associations}
\label{sec:mod-com-sec-assoc}

Model MS is à structure aspect of the IS.~We suppose that IS is implemented in object"=oriented programming languages or, at least, in languages supporting some kind of objects (abstract data types) in sense of data encapsulation, е.g., structure programming. MS defines the attribute structure and operations on the objects. Objects have associations with other objects, and the association also defined in MS.~The Class diagram of the UML"=language is à way for MS formalization. According to UML specification the objects are represented by Class classifier, which, in general, define пате of the object, list of its attributes, and operations on the object. UMLassociations define the relation between objects.

Class diagram can be used to define object structure for almost any kind of software, but in our case the common techniques of the IS design as an enterprise IS should be accounted, е.g., attribute data are usually stored in a relational database. Last means that some additional information associated with the attribute, like attribute пате of à table, where the attribute will be stored. The UML"=language supports à possibility to associate the additional infoimation with any element of Class diagram as well as to define new kinds of classifiers. In the first case tag values are used, е.g., tag value display name can associate string "First name" with attribute first name of class Person. In the second case Ъу means of stereotypes one can define some peculiar properties on Classes, е.g., it can be pointed out that the attributes of Person instances are stored in PERSON table of à rational database. Thus, tag values and stereotypes are used to extend Class diagram with specific information about the IS as à special case of programming system.

\subsection{Semantic interpretation of components and associations}
\label{sec:sem-int}

Generated source code of program aspects is understood as the semantic interpretation. The semantic interpretation is implemented by means of knowledge base and source code templates. Knowledge base contains formalized qualitative characteristics of relations between structural compositions of Class diagram elements, as well as associations to source code templates in the context of an environment defined with tag values and stereotypes and target implementation platform. In the process of the compositions recognition some intermediate conclusions are done, specifying the PSM.

Let us consider an example. For à Class marked as \texttt{<<abstract>>} stereotype usually it is not necessary to generate SQL statement of table definition, as for abstract class has no instances. The SQL's PSM will not contain this class. All the attributes of the class are inherited by descendants, i.e., moved to corresponding descendant table definitions. А SQL view is generated to gather all the descendant class instances for à possible query. Thus, according to Class diagram the generating system creates an abstract class for business"=logics program aspect (business logics PSM has an reflection of the class), but creates no corresponding БЯЬ4аЫе definition. Thus, when à composition of elements has been recognized one of two actions can happen: à fact derivation, which is stored for further usage; source template filling in derived term values and resulting source code generation.

\subsection{OCL constraints}
\label{sec:sem-OCL}

Model of constraints is an important part of the IS's PIM model. The following program components are examples of constraint implementations: atomic data types, value ranges, input patterns, database triggers, CONSTRAIN definitions in nowadays SQL.~Some elements of Class diagrams can have soundness check constraints, which should be checked by database engine in the case of table modification. UML specification contains the ОС1 language (Object Constraint Language) specification. The OCL language is useful to define constrains formally. Defined constrains are translated to their source code implementations in program aspects.

Consider the following example. The text "Registration time of à person is earlier than departure time" could be translated to ОСЬ in some Person object context as follows:

context Person inv:
self.reg time < self.dep time
The ОСЬ-expression is an invariant, i.е., holds always
during life time of the object after construction.
The expression is translated to the following source
code:

- for data warehousing program aspect (relational database):

ALTER TABLE PERSON
ADD CONSTRAINT
REG TIME < DEP TIME

- for user interface program aspect (а JavaScript validator within à HTML"=form):

if (!this->reg time<this->dep time) { alert(...some warning text...) };

The warning text is à text associated with the constraint.

ОС1 supports language structures which can be translated to à complex SQL"=queries. Triggers are generated by means of analyzing combinations of pos t:- and pre: -conditions and pre9"=prefixes. The ОСЬexpression are translated in context of MI, also specifying PSM.

\section{Logical MDA implementation}
\label{sec:log-MDA-impl}

Figure 1 illustrates system architecture and interaction of modules of the program generation software.

\begin{figure}[htb]
  \centering

  \caption[Generator architecture]{System architecture and interaction of modules of the program generation software.}
  \label{fig:gen-arch}
\end{figure}

\subsection{Architecture of generation software and its basic principles of functioning}
\label{sec:basic-func}

The UML Class diagram is defined by visual UML editor, е.g., PoseidonCE \cite{b4}. XMI"=translator is à standard DOM2"=compliant translator, we use à Python DOM2 parsing library (руХМЬ"=0.8.3). The IS PIM Model as object net is conducted by XMIParser.py module opensource library ArchGenXML \cite{b5}, during the conversion OCL"=expressions are extracted and parsed into à syntax parsing trees. The obtained object net is converted to à set of atom predicates of the Prolog language. Thus, the Class diagram during code generation is represented by means of three data formats.

The diagram is sent to the top of à generation module net. Each module recognizes some aspect of IS's design and makes deductions as PSM specification or generates a piece of source code as the constructed PSM implementation. The topmost module make general decisions like whether to consider an attribute as possessed by an object, converts object structure in general. The obtained decisions transferred to other generator modules.

The resulting source code is generated by the lowest modules. Each generator module uses its set of rules, which are stored in the module, so the knowledge base is distributed among the modules.

\subsection{Generation module implementation}

Each module consists ой~чо parts: recognition template and action. Recognition template is à Prolog rule or set of Prolog rules. If à query of the rule is successful for a number of objects, the action parametrized by the objects is initiated. Action can add à fact to the general fact list or generate à piece of source code.

Generation modules are realized as Python language \cite{b6} instance methods of à class. The Prolog rules are written in so called docstrings of the methods. The docstrings are defined by special Python syntax constructions and accessible as doc -attribute of methods at runtime. Query to à generator is done by an wide"=class API"=method query, which is à generator. The set (tuple) of objects, on which the query is successful, can be also caught by a for statement. So, for the successful query not only module action is initiated, but some action defined by the f or"=statement.

Consider an example of module construction for generation à SQL"=table for à nonabstract class.

class Ru1esMixing:
№ ruleset for SQL table generation
def rule generating class(se1f, c1s):
№ the class attributes are stored in à table,
№ if the class is not abstract,
¹ marked by «OODB» stereotype
element (C1s, ' Class ' ),
3+stereotype(C1s, 'abstract')
яtereotype(Cls 'OQDB'),
3+internal only(Cls).
def rule c1ass~arent(se1f, c1s, parent)
¹ Parent is one of the nonabstract
class~arent(C1s, Gparent):—
№ this two modules have no acti
lass SQLTranslator(Translator, Ru1esMixing)
№ SQL-script generation module set

def genClass(self, cls)
answer. append(attribs)
№ SQL-script generator for class cls
answer = (l
№ the source code is à list of lines
if cls in self. generated:
№ if the class is already generated
return answer № do nothing
for , parent in
self. query('c1àss рагеп~'
(cls, '№') ):
№ generate all nonabstract parents
name=self.getName(cls) № class id
at.tribs = self.genSchema(cls)
№ generate àttributes of class
answer. append("CREATE TABLE %s (" %
name)
answer. append(")%s;" %
self.getTableType(cls))
self.addFact("oodb table('%s', '%я')" %
(cls getId(), name))
№ adding the fact of reflection of
№ the attribute to
№ à corresponding SQL table field
self. generated. append(cls)
№ mark the class as already generated
return answer № return the source code

\section{Logical inference}
\label{sec:log-inf}

\begin{figure}[htb]
  \centering

  \caption{Test figure}
  \label{fig:test-fig}
\end{figure}

\begin{table}[htb]
  \centering
%  \begin{tabular}[1]{11}

%  \end{tabular}
  \caption{Test Table}
  \label{tab:test-tbl}
\end{table}

\section{Theoretical considerations}
\label{sec:theor-general}

MDA as an intelligent configurator for generative programming

Component Architectures

\section{Alternative approaches}
\label{sec:alt}

Generic programming (Julia, D, Rust)



\section{Conclusion}

An approach to implement code generators within OMG's MDA architecture is considered for à special class of enterprise information systems. А technique and software for generation of sound source code of main subsystems from one formal MDA model is developed. Under the soundness we consider the appropriate reflection of an element of UML Class diagram (а part of the model) in the generated subsystems.

The software is based on à system of artiflicial intelligence, namely à knowledge based system. The code generation result from an inference on initial set of facts, representing as à specially enriched UML Class diagram. The generator system supports various levels of abstraction of the information system representation. The levels are defined by set of code generators. This allows one to transit from one implementation technology to another by means of realizing new generator subnet. On each level an intermediate (in sense of abstraction) variant of a Platform Specific Models (PSM) generated, which is specified on the lower levels. Thanks to the feature one can construct à net hierarchy of PSM models, allowing one to generalize design knowledge among à set of common implementation technologies.

\subsection{Further pushdown}

One of further directions of the project development is to involve in the modeling process other classes of UML diagrams and other OMG's standards, висЬ as the MetaObject Facility (MOF), and the Common Warehouse Metamodel (CWM). The another direction is to construct PIM model version difference analysis system. On the base of the analysis, е.g., à SQL script can be generated, which implements à transition of old database structure to new one.

Aggregation and reports

\section{Future Works}

Software development life cycle has been considered as subject of the theory of complex systems and complexes [4] implying that the software development is a natural process. The life cycle is represented as system of models and morphisms between them. Analysis of the theory’s properties realization in the model shown, that the present instrumental software productivity could be extended by means of developing techniques for analysis of the passed life cycle stages, analysis and difference propagation of the models.

To support the propagation of the differences existing examples of the present source code development technologies should be adapted. Namely, known patchfile format is considered and some its modification is suggested to account properties of generated source code. An approach to realization of the difference propagation is also briefly considered. It is shown, that the transformation technique and file formats are closed with respect to original formats of the software models representation.

One of the aims of the research is to constructsoftware development tools based on analogy. For example, having stores in a revision control systems all the states, models and stages of MDA software development as the differences, it probably be possibleconstruct new sequence of differences for new original model.

To gain some new experience about the topic a pilot project started to automate document preparation innotarial office, where users have a dominant roledeveloping the software function set.

\section{Acknowledgment}

The research is partially supported by the Council for grants of the President of Russian Federation, state support of the leading scientific schools, project No NSh-8081.2016.9.  The presented software is developed using the Shared Computing Center of Integrated Information and Computing Network of Irkutsk Research and Educational Complex \url{http://net.icc.ru}.


\section{REFS}
\label{sec:refs-tmp}

[1] T. Curran, G. Keller, А. 1.а44, SAP R/3 Business Blueprint:
Understanding the Business Process Reference Model,
Prentice Hall PTR, USA, 1997.

[2] G. Booch, J. Rumbaugh, 1. Jacobson, The Unified Modeling
Language User Guide, 1st edition, Addison-Wesley Pub
Со., Boston, 1998.

[3] OMG Model Driven Architecture. http: //www.omg.org/mda/

[4] Poseidon for UML - by Gentleware, just model — Ноте.
http: //gentleware.corn/

[5] ArchGenXML Manual - generating Archetypes using UML.
http: //p lone. org/documentation/archetypes/archgenxml-
manual/

[6] Python Programming Language. \url{http://www.python.org/}

MDA2003

kuzmin

italian disser

find something new

logtalk http://logtalk.org/documentation.html A Portable and Efficient Implementation of Coinductive Logic Programming. Paulo Moura. Proceedings of the Fifteenth International Symposium on Practical Aspects of Declarative Languages (PADL 2013), 2013. Springer LNCS 7752. \url{http://dx.doi.org/10.1007/978-3-642-45284-0_6}

\end{document}

%%% Local Variables:
%%% mode: latex
%%% TeX-master: t
%%% TeX-command-extra-options: "-shell-escape"
%%% End:
