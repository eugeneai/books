\documentclass[draft,openany,14pt]{extbook}
%\usepackage[qa,fancytop,ptfonts]{subook}
\usepackage[times,fancytop,ptfonts,a4paper]{subook}
\setlength{\parindent}{3.5ex}
\begin{document}
%\setlength{\itemsep}{0pt plus 0pt minus 0pt}
%\setlength{\parsep}{0pt plus 0pt minus 0pt}
%\itemsep0pt
%\parsep0pt

\begingroup
\thispagestyle{empty}
\mbox{}

\ptsans
\vspace{13ex}
\centering\large
\noindent{}A.~K.~Cherkashin

\vspace{6ex}

{\Huge\bfseries

POLYSYTEM

ANALYSIS

AND SYNTHESIS

}
\vspace{1em}

The application in geography
\\[0.7em]
{\normalsize{} Editor {\itshape{} V.~S.~Mikheev}}
\vfill{}

Irkutsk 1997
\mbox{}

\endgroup
\newpage
\begingroup
\mbox{}
\thispagestyle{empty}

\ptsans
\vspace{13ex}
\centering\large
\noindent{}А.~К.~Черкашин

\vspace{6ex}

{\Huge\bfseries

ПОЛИСИСТЕМНЫЙ

АНАЛИЗ

И СИНТЕЗ

}
\vspace{1em}

Приложение в географии
\vspace{1em}

{\normalsize\parskip0pt
Ответственный редактор\\[-0.5em] доктор географических наук {\itshape В.~С.~Михеев}}
\vfill{}

Иркутск 1997
\mbox{}

\endgroup
\newpage{}
\begingroup
\newcommand\sucopyright{{\ptserif\copyright}}
\newcommand\ISBN{ISBN 5--02--030607--X}
\thispagestyle{empty}
\noindent{}УДК 167:510:910

\noindent{}ББК 22.172

\vspace{2em}

\noindent{}\begin{minipage}[t]{3em}
\noindent{}Ч 48%
\end{minipage}\hfill{}%
\begin{minipage}[t]{0.90\linewidth}
\setlength{\parindent}{3.5ex}

Черкашин~А.~К. {\bfseries Полисистемный анализ и синтез. Приложение в географии}\,{}/
А.~К.~Черкашин.~--- Новосибирск: Наука. Сиб. предприятие
РАН, 1997.~---~\pageref{LastPage}~с.\\[0.3em]
\ISBN{}.
\vspace{1em}

\begin{abstract}\small
В книге исследуется сложный объект (планета Земля и свойственные ей процессы и явления) с разных сторон, в различных системных проекциях с использованием полисистемной методологии, возникающей из логики анализа противоположностей, понятий и законов общей теории систем в авторской интерпретации и из современных математических представлений о многомерных расслоенных пространствах. Для каждой научной проекции разрабатывается аксиоматическая теория, описывающая на специальном системном языке сквозным образом природные, экономические и социальные структуры и их изменения. Выделено около 20 теоретических направлений, отражающих различные аспекты изучения географических явлений разного масштаба, упорядочивающих имеющееся знание и позволяющих получить новые объяснения фактам. Для каждого направления дается краткий анализ понятийной базы, излагается используемый математический аппарат и на примерах иллюстрируются полученные выводы.

Книга представляет интерес для исследователей, работающих к различных областях науки и интересующихся проблемами поиска оснований формирования теоретических знаний и развития системной методологии решения конкретных задач.

Табл. 8. Ил. 94. Библиогр.: 563 назв.
\end{abstract}

\vspace{2em}
\centering
\textbf{Рецензенты}

доктор географических наук \textit{Ю.~М.~Семенов}

доктор физико-математических наук \textit{Г.~Н.~Константинов}
\vspace{1ex}

Работа выполнена при финансовой поддержке

Российского фонда фундаментальных исследований.

Проект 93-05-14076

"Теория и методы полигеосистемного анализа".

Проект 96-05-64727

"Геоинформационные процессы: исследование и моделирование"
\end{minipage}

\vfill{}
\noindent{}\begin{minipage}[t]{10em}
\mbox{}\par
\ISBN{}
\end{minipage}\hfill
\begin{minipage}[t]{0.51\linewidth{}}
\noindent\sucopyright~А.~К.~Черкашин,~1997\par
\noindent\sucopyright~Российская академия наук, 1997
\end{minipage}
\endgroup
\newpage
\tableofcontents{}
\chapter*{Предисловие}
Логической базой всей географии была выработанная к концу XIX в. (А. Гумбольдт, В. Докучаев и др.) особенность изучения целостной картины природы, с помощью которой для осмысления естественных отношений использованы цельные зрительные образы неопределенной формы, в точности соответствующие пониманию природных объектов как "многого в едином". Поскольку эти образы были зрительными, они картографировались, поскольку находили отражение в конкретных явлениях, то выражались через природные компоненты и их связи, что способствовало развитию отраслевых разделов географии. На теоретические основания классического ландшафтоведения (начало~--- середина ХХ в.) были возложены надежды на установление и описание всеобщей связи явлений и их объективной субординации. Началась онтологизация понятий, обеспечивающих целостность восприятия, что содействовало быстрому становлению теоретических основ современной географии, в частности аппарата исследований для получения новых фактов и проверки истинности научных выводов, что привело к конкретным результатам, по-новому характеризующим природные и социально-экономические образования, например в учении о зональности. Тем самым был создан необходимый фундамент для перевода многих представлений географии из разряда интуитивных в разряд более точных.

В последние годы по ряду объективных причин география решительно перешла от визуальных наблюдений и обобщений к научно организованным исследованиям в контролируемых условиях. Она берет на вооружение и разрабатывает методические подходы, основанные на применении современных приборов и новых технологий для решения комплексных географических проблем. Необходимо понять, как вовлекаются в действие различные структуры географических объектов, как их анализировать: по отдельности, с последующим изучением их взаимосвязи или сразу, взятые в целом? На этом пути необходимо, преодолевая инерцию внутригеографического мышления, найти общенаучный подход к теории и практике операций с комплексными объектами. Формируется тенденция, предъявляющая возросшие требования к метатеории географии и выявившая несоответствия между имеющимися гносеологическими принципами и новыми научными фактами, что привело если не к дискредитации наивной теоретико-познавательной основы классической географии, то к известной неудовлетворенности ее постулатами и пониманию необходимости их замены более строгими положениями.

Монография доктора географических наук А.~К.~Черкашина в этом смысле имеет особый характер. Это оригинальный труд ученого, который, обладая большим личным опытом исследовательской работы в различных областях знания, стремится сформулировать новые задачи, возникшие перед географической наукой, и наметить пути и методы их решения. В центре внимания автора~--- теория и методология полисистемного анализа и синтеза, значение которых читатель начинает понимать только тогда, когда он воспринимает явления и объекты географической реальности как сложные образования, которые невозможно описать в рамках одной теории. При этом полисистемный анализ выступает как специальное направление конструктивного исследования сложных объектов. Его методы исходят из теоретического и методического базиса общей теории систем, формальной и диалектической логик, успешно применяемых разными науками. Идея полисистемности строения природы, а вместе с ней созданных человеком социально"=экономических образований, на фоне недостаточности информации о их свойствах, процессах и состояниях вносит новые элементы в системные исследования географической среды. Во"=первых, возникает необходимость в ином подходе к классификации системных качеств, методов их описания, выделения и закрепления в научной практике. Во"=вторых, что принципиально, в системных исследованиях при реализации принципа объективности выделения систем разного рода возникает информационный барьер между качественной и количественной соизмеримостью неоднозначных, неравноценных и недостаточных фактов, поэтому содержательное изучение определенного типа систем необходимо дополнять совокупностью исследовательских средств и методов, обеспечивающих достоверность и упорядоченность информации в системе знаний. В"=третьих, требуется новый подход к созданию информационной базы изучения природного целого и ее формализации для описания зависимостей между элементами и связями системы.

Способы получения и использования знаний разрабатываются на разных уровнях. В географии созданы и продолжают развиваться исследования, основанные на методах сбора информации с последующим эмпирическим обобщением данных. Естественно, они никогда не достигнут стадии полного завершения, так как любой опыт не может быть интерпретирован в абстракции от понятийного и логического аппарата, который делает возможным научную интерпретацию. Сам факт чувственного восприятия предполагает активность нашего сознания и является продуктом не только одних "чувственных сигналов", как например у Э.~Маха, но и размышления, принимающего в расчет многоступенчатый критерий практики и учитывающий не только факты реальности в "наивном", несистематизированном и логически не оформленном понимании, но и их связь.  Сознание проделывает большую селекционную и классификационную работу, что приводит к необходимости находить новые, обновлять традиционные или приспосабливать заимствованные из разных наук методические приемы, процедуры и исследовательские операции. Именно систематизации методов, определенной перестройке оснований географии и ее постепенной строгой формализации, как это было в других точных науках, например в физике и математике, явившихся ориентиром создания новых познавательных средств, и посвящено исследование А.~К.~Черкашина.

Им подняты важные вопросы обобщения опыта получения знаний на этапе перехода от эмпирического к теоретическому уровню с использованием процедур и алгоритмов разных видов исследовательской работы. Речь идет не просто о необходимости выделения и проведения познавательных операций над наблюдаемыми процессами и явлениями природы, а операций над комплексными объектами, которые не могут быть объяснены и предсказаны, опираясь только на "здравый смысл", они требуют создания сложных и многоступенчатых понятийных теорий. Суть заключается в том, что научный географический анализ на первых порах был "наукой о природе", где нерасторжимо слиты естественно"=научные факты, выработанные человеческим интеллектом,~--- интуицию и логика, проникшие из естественного языка самой природы в описательную географию. Изучение логических аспектов анализа не было актуальным. Теперь положение изменилось, поскольку, имея дело с миром сложных систем, географам стало трудно сверять такую теорию "непосредственно с вещами материального мира" и требуется своеобразный компас, позволяющий не сбиться с пути и вести исследования правильным способом. На роль такого компаса все больше начинает претендовать строгий логический метод рассуждения, в результате чего потребность в изучении и применении законов логики значительно возросла. Полисистемный подход уделяет особое внимание моделированию сложных объектов природы и использованию логических и математических методов. Этой важной стороне объективного развития географической науки в работе А.~К.~Черкашина отведено значительное место.

В поле зрения автора находится проблема взаимодействия и коммуникации различных областей знания, чем, в частности, объясняется большое внимание к моделям и понятийно"=терминологической базе исследований. Такой подход применен для формирования системы понятий и показа широких возможностей оперирования элементами языка теории~--- аксиомами, терминами, определениями~--- по четким правилам логики с обоснованием достоверности полученных результатов. Автор обобщил методы формализации многих наук, поставил их знания "на службу географии", чем выразил давнее стремление географов осознанно отражать факты в понятиях более широкого охвата при описании и выделении разных природных и социально"=экономических образований. По существу, им представлена миогоаспектная программа формализации географических законов и закономерностей для выделения генеральных сквозных направлений теоретического обобщения результатов исследований и особенностей методических действий при решении конкретно"=научных проблем, а также показаны возможности их применения в географии и других науках. Посредством таких представлений техника моделирования и математические модели естественным образом соединяются с содержательной географией эмпирических обобщений.

Конечно, следует отдавать себе отчет в том, что научная проблематика, о которой идет речь в книге, очень сложна, многогранна и еще недостаточно разработана, поэтому поворот современной географической науки в эту сторону еще далеко не завершен и дает, как это видно из исследования А.К. Черкашина, первые полезные результаты. Рассмотрение им проблемы содержат элементы неопределенности и дискуссионности. Познание комплексных природных и других объектов как сложных полисистем~--- это процесс более трудный и многоступенчатый, чем было принято думать раньше. Он связан с построением миожества вспомогательных научных понятий, не имеющих прообразов в реальной природе и познавательных ситуациях, но дающих возможность перейти от общих философских категорий к конкретным уравнениям и расчетам, показав логические истоки возникновения каждой закономерности.

Исследования А.К. Черкашина позволяют развить экстенсивное представление о структуре географического исследования через обоснование существования уровней и типов исходной информации, получаемой из различных. источников, которые, будучи обобщенными едиными методами, сводятся в итоге к эквивалентным интерпретациям научных знаний. Это восполняет пробел между традиционным изучением географических явлений и географией, идущей ему на смену.

\begin{flushright}
\small{}
Доктор географических наук\par{}
\itshape{}
В.~С.~Михеев
\end{flushright}

\chapter*{Введение}

Основная проблема современной науки~--- это проблема понимания сложного. Жизнь предоставляет достаточно фактов, чтобы убедиться, что любое сложное явление не становится простым по мере расширения наших познаний, а, напротив, превращается в еще более сложное и непонятное. Поэтому сложность как универсальное качество объектов должна рассматриваться не относительно нашего уровня знаний, а считаться в такой же степени реальной, как и другие фундаментальные свойства объектов. Такой подход требует от исследователей соответствующей методологии и логики мышления, новых взглядов на устоявшиеся представления о процессах и явлениях.

В науке есть отрасли знания, для которых работа со сложными объектами, явлениями и процессами~--- основа существования их предмета исследований. Это древние, так называемые синтетические отрасли, к которым прежде всего следует отнести историю,медицину и географию. Известная латинская пословица~--- historia est mater studiorum~--- отражает суть исторического познания как начала отсчета всякого человеческого знания.

Целостному синтетическому восприятию сложных процессов и явлений мешает несколько причин. Во"=первых, не ясно, что нужно синтезировать. По"=видимому, дифференциация научного знания еще не достигла той степени детализации, чтобы увидеть кирпичики научного знания, из которого строится целостная картина мира. Во"=вторых, не установлены принципы системного синтеза. В"=третьих, логика и методология, необходимые для этого анализа и синтеза еще недостаточно совершенны.

Исторически логика современной науки была заложена более 2~тыс. лет назад в работах Аристотеля по формальной логике. Именно она стала идеалом научного мышления и прежде всего в математике и физике, что в большой степени обусловило их постоянный прогресс. Современная компьютеризация~--- явное выражение триумфа формальной логики, поскольку вычислительные машины работают исключительно на принципах свойственных ей бинарных исчислений.

Однако логика жизни и человеческого мышления выходит далеко за пределы формальных бинарных алгоритмов, что особенно хорошо показала разработка экспертных систем. Содержательные знания экспертов с трудом удается реализовать в компьютерных программах, а то, что получается в результате, намного уступает возможностям эксперта.

Осознание ограниченности формальной логики привело к созданию ряда новых логических систем, среди которых выделяется диалектическая логика~--- трудная для понимания, но наиболее приближенная к реальности логическая система. Как известно, диалектическая логика научно оформилась в работах Гегеля, но не получила распространения как средство прикладного научного анализа особенно с использованием аксиоматического метода, который позволяет выводить знания из ограниченного количества базисных понятий и законов.

Для того чтобы с помощью теоретических средств понять идею сложного, необходимо идти парадоксальным путем~--- через углубление дифференциации науки и соответствующих знаний о мире.  Этот путь приводит нас к разработке концепции полисистемного анализа и синтеза. Сущность полисистемного анализа состоит в естественной возможности разделить, расслоить все представления о сложных наблюдаемых объектах на систему непересекающихся множеств понятий и законов. Расслаивая, мы как бы проецируем природные объекты или знания о них в каждое из подобных множеств, представляя единое в системе координат их частных отображений (слоев). Задача полисистемного анализа~--- обеспечить процедуру расслоения и определить на совокупности слоев"=отображений структуру или систему отношений между слоями. Обычно они задаются отношениями взаимного отображения (сравнения, тождества), комбинациями разных слоев, различного рода морфизмами, оформленными в коммутативные диаграммы и др. Это реализуется средствами старой диалектической логики и с помощью новейших процедур абстрактной алгебры.

В рамках такого подхода полисистемный анализ предлагает новые методы систематизации и представления знаний: после специального структурирования и глубокого расслоения отдельных отраслей знания соответствующие им аксиоматические системы (теории) могут быть построены по образу и подобию общей теории систем или диалектики. Это достигается путем интерпретации (соответствующей замены) понятий одной теории на понятия другой. Реализация такого подхода дает возможность получить необычные аксиоматические системы для отраслей знания, которые раньше ими не обладали.

Решение этой проблемы имеет методологическое значение для всех наук. Так, известный физик P. Фейнман \cite{b437} писал, что "сегодня наши физические теории, законы физики~--- множество разрозненных частей и обрывков, плохо сочетающихся друг с другом.  Физика еще не превратилась в единую конструкцию, где каждая часть на своем месте". Зги слова все еще актуальны, как и многие проблемы оснований математики Д.~Гильберта, сформулированные 100 лет назад \cite{b347}. Среди них есть шестая проблема, касающаяся непосредственно естественных наук\,: "\ldots{}с исследованиями по основаниям геометрии близко связана задача об аксиоматическом построении по этому же образцу тех физических дисциплин, в которых уже теперь математика играет выдающуюся роль\ldots{}" (Там же, с. 34). На полвека позже Л.~фон~Берталанфи, создавая общую теорию систем, также мечтал о возможности по ее образу и подобию конструировать теории других наук (см.\,: \cite{b367}).

С аналогичными проблемами сталкивается такая древняя наука, как география. Возникнув как описательная, она лишь с середины XIX в. начала оформляться как относительно самостоятельная, синтетическая область знания о связях и взаимодействии тел и явлений земной поверхности [118, 176]. Дальнейшая ее дифференциация привела к ситуации, сходной с той, которая сложилась в физике и других науках. Академик В.Б. Сочава почти фейнмановскими словами писал\,: "Географические науки, вместе взятые, пока что не образуют целого, состоящего из органически сочетающихся частей, или совокупности взаимосвязанных и расположенных в определенном порядке элементов" [397, с. 480]. Для преодоления этого состояния В.Б. Сочава разработал учение о геосистемах и определил основные задачи и направления исследований. Первым в ряду этих проблем указан "анализ аксиом и других положений специальной теории геосистем как частей общей теории (метатеории) систем" [397, с. 1S].

Теоретизация географии прежде всего требует выделения объекта и предмета этой науки. Здесь мы будем исходить из известных представлений А.~Гумбольдта об этой науке как комплексной дисциплине. Ярким представителем комплексного подхода в науках о Земле был В.В. Докучаев. В его программе"=проекте детального изучения Санкт"=Петербурга и его окрестностей (1890~--- 1892 гг.) сформулирован общий принцип «одновременного, цельного всестороннего исследования "определенного района", природу которого необходимо изучать, взятую в целом, единую и нераздельную\ldots{}» [1S2, с. 461].

Концепция В.~В.~Докучаева, поставившего в основание географии представления о целостности взаимосвязи между природными комплексами, получила дальнейшее развитие в учении о ландшафте (Л.~С.~Берг, Б.~Б.~Полынов), в котором виделись черты будущего "синтетического естествознания" и пути его практического применения в связи с задачами освоения природных ресурсов. В изучении комплекса явлений во внешней оболочке Земли заключается специфика ландшафтоведения и его несводимость к одной только предметной области изучения физических тел природы или проявления физических, химических, биологических и других законов [2BS]. Со становлением теории ландшафтоведения (Н.~А.~Солнцев, Ф.~Н.~Мильков, А.~Г.~Исаченко, Д.~Л.~Арманд и др.) получили развитие и общие вопросы методологии ландшафтных исследований \cite{b288}, что в различных интерпретациях раскрывается в работах многих ученых. На синтетической роли географии по отношению к смежным отраслям знания акцентирует внимание в своей монографии П.~Хаггет \cite{b450}. Можно говорить о становлении, как образно отметил А.~А.~Крауклис [227, с. 206], своего рода "географической формы точности" в научных исследованиях. При этом географический объект исследуется всякий раз как явление уникальное, и приоритет уникальности при хозяйственном освоении территории и специализации человеческой деятельности становится важнейшей предпосылкой планирования и проектирования \cite{b351}.

\begin{figure}
\caption{Геометрическая модель, поясняющая принципы формирования закономерностей или сложных ситуаций как пересечения множества законов в обобщенном пространстве признаюзв географических объектов (по\,: \cite{b24}).  а--- линия"=символ $M$---$M$' отдельного закона природы, представляющего единство двух независимых начаи $M$ и $M'$; б и в~--- две разные закономерности, являющиеся пересечением или наложением нескольких самостоятельных законов ($P$---$P'$, $Q$---$Q'$, $N$---$N'$ и т.д.).}\label{pic:intro:1}
\end{figure}


Понятия индивидуальности и конкретности хорошо иллюстрируются представлениями о закономерности (сложной ситуации) (рис. \ref{pic:intro:1}). Всякое событие, явление, процесс~--- есть стечение обстоятельств, пересечение различных законов. В этом смысле закономерность, или сложный закон,~--- это точка пересечения пучка прямых, каждая из которых соответствует закону взаимодействия противоположных начал. Такая модель определяет некоторый общий принцип и символ географического знания и мышления.

Проблема вычленения предмета и объекта географии неоднократно обсуждалась в специальной литературе [98, 277, 300]. B большинстве случаев предмет географии отождествляется с ее объектом~--- географической оболочкой Земли \cite{b284}. В соответствии с определением, приведенным в "Философской энциклопедии" \cite{b441}, под предметом исследования обычно понимается та сторона, то свойство объекта, которые рассматриваются в данном исследовании. Отмечается, что при таком подходе "один и тот же объект может быть предметом ряда различных исследований" (с. 357), и здесь возникает особая проблема синтеза этих различных предметов при построении единой теории объекта, т.е. такой теории, предметом которой является сам объект. Таким образом, возможна теория, в которой предмет и объект как противоположности отождествляются.

Проблема синтеза отдельных предметных свойств в единый объект родственна задачам, которые призвана решать география. Поэтому ее с полным правом можно отнести к разряду тех наук, предметом изучения которых является сам объект в его единстве. Вместе с тем для географии свойственно внутреннее предметное расслоение.  Еще О.~А.~Константинов \cite{b207} обратил внимание на то, что "все географические науки (кроме физической географии) одновременно принадлежат к соответствующим отраслевым наукам. Так, фитогеография и зоогеография являются также науками и биологическими, климатология и гидрология~--- физическими, историческая география является также наукой исторической \ldots{}" (с. 107). Все это сказывается на методическом разнообразии средств и форм географических исследований и закономерно приводит к пониманию существования сквозных направлений в географии (по терминологии К.~К.~Маркова [71, с. 47]), а следовательно, и разных сквозных теоретических интерпретаций знаний о географической оболочке.

Вместе с тем география не одинока в синтетическом, конкретном, многостороннем отражении окружающего нас сложного мира.  Зги же проблемы волнуют историков [30,162], медиков \cite{b464}, представляющих науки, в которых возникает и решается проблема синтетического человекознания [14, 271]. B технике при конструировании также появляется потребность в интеграции, синтезе, рассмотрении различных сторон явлений \cite{b326}. В совокупности эти науки образуют сквозное поле синтетического знания.

С другой стороны, спецификой теории геосистем, объединяющей знание отраслевых географических дисциплин, является изучение структурно"=динамических свойств географических объектов разного ранга и размера, обусловленных разнообразными взаимосвязями компонентов природной среды. Такие же задачи, и в этом легко убедиться, по отношению к своему объекту решают статистическая физика и химическая кинетика, физиология животных и растений, экология и экономика, технологические дисциплины и конкретная социология. В совокупности они образуют сквозное поле знаний другой (динамической) системной интерпретации свойств объектов. B названиях перечисленных наук хорошо просматриваются объекты"=аналоги из поля синтетического знания, например сопоставляются социология и история как разные научные интерпретации объектов социальной сферы.

В работе на доступной теоретической и фактической базе обосновываются следующие положения\,:

\begin{enumerate}\bfseries\itemsep0pt \parsep0pt
\item Современный мир устроен так, что вся совокупность знаний о нем расслаивается на множество непересекающихся сквозных областей, для каждой из которых существует своя полная теория представления знаний.

\item Все теории отличаются базовыми понятиями и подобны друг другу через интерпретацию (замену понятый), что позволяет индуцировать новые аксиоматические теории по образу и подобию известных.

\item Каждый системный слой знаний многократно и последовательно расслаивается, образуя полисистему представлений о любом объекте от вселенной до элементарных частиц.

\item Все системные теории объединяются в единую науку, описывающую один из возможных миров; знания этих теорий являются основой для синтеза целостного представления о наблюдаемых объектах.
\end{enumerate}

Очевидно, что в такой системе знаний географическая наука в силу ограниченности объекта своего познания (ландшафтной оболочки Земли) в пространстве и во времени~--- лишь фрагмент, позволяющий через решение своих проблем понять весь мир. Географическое знание подобно знаниям других современных наук расслаивается, проецируется в различные теоретические системные слои.  Поэтому существует не одна, а множество (полисистема) географий, в которых особо выделяется комплексная география как раздел сквозной теории о комплексах.

\part{Методология исследований}
\chapter{Принципы полисистемного расслоения}

Основные результаты полисистемного анализа и синтеза вытекают из представлений о расслоении свойств объектов. Процедура расслоения и связанные с ней методы работы со слоями состоят из нескольких этапов и приводят к разным результатам в зависимости от характера базы расслоения и решаемых задач. Однако они имеют достаточно общий характер и основываются на единой системе понятий и методологии исследований. Первоначальное знакомство с этими понятиями и отношениями задает своеобразную матрицу восприятия всего последующего материала независимо от его особенностей.

\section{Предпосылки формирования полисистемной методологии}

Исторически формирование полисистемной методологии восходит к работам философов и математиков, а также географов, одними из первых увидевших в ней перспективные методы исследования сложных природных и природно"=экономических систем. Современное представление о поли- и моносистемах концептуально и терминологически связывается с работами В.~П.~Кузьмина [232, 233]. Он исходил из анализа представлений К.~Маркса о двойственном характере труда и развил идею множественности системных оснований различных системных качеств и их единства как полисистемы. В основе лежит философский принцип единства и тождества противоположностей в форме, восходящей к работам Гегеля и его предшественников. В другом контексте аналогичные идеи независимо излагались Ю.~А.~Урманцевым [428, 429, 432]. Он также писал о множественности систем различного рода и предложил метод композиционной классификации их элементов, который с успехом используется в различных областях науки \cite{b381}. По принципам полисистемного представления действительности важные теоретические и практические результаты получены при изучении мышления и мыследеятельности [101,512 и др.], к числу которых относится схема многоплоскостной организации знаний об объекте.

Понятия "моносистема" и "полисистема" введены в географию В.~C.~Преображенским \cite{b338}. Они отражают двойственную природу географических объектов, выражающуюся в подразделении объектов соответственно на типологические и региональные единицы.  В.~Б.~Сочава [398, 399] постулировал принцип двухрядной классификации геосистем как основополагающий при их изучении. Моносистемы~--- геомеры~--- внутренне однородные, гомогенные пространственные образования; полисистемы~--- геохоры~--- разнородные, гетерогенные пространственные структуры, состоящие из геомер. Тем самым полисистемность вошла в географию через задачи типологии и классификации. Именно здесь развивается идея полигеосистемного анализа как сформировать совокупности специальных средств и методов проективного расслоения сложного географического объекта на множество системных представлений (интерпретаций) через отображение его свойств в разных предметных областях с установлением структуры отношений между этими свойствами.

Появление полисистем в математике связывается с процедурой расслоения, т.~е. разбиением математического объекта на множество непересекающихся подмножеств. Трудно найти область математики, в которой бы этот метод в той или иной интерпретации не использовался. Наивысшее выражение он получил в теории математических категорий и других разделах универсальной алгебры [57, 123, 325, 335]. Суть подхода заключается в том, что любой объект можно расслоить и тем самым представить его в многомерном пространстве своих слоев"=координат. Такой подход используется при решении задач управления [11, 12, 252], декомпозиции, анализе и синтезе сложных систем \cite{b391}. Полезность этого абстрактного математического аппарата для науки на примере физики хорошо показана И.~Л.~Герловиным \cite{b104}.

В отличие от системного подхода, где принцип связи элементов является основополагающим, полисистемный анализ исходит из гипотезы расслоения, т.~е. верит в возможность представления объектов как непересекающихся множеств, а значит, несвязанных между собой слоев. В этом смысле всякая полисистема~--- антисистема, отношения элементов"=слоев которой задаются через отображения, а не через контактные связи. Но в обобщенном понимании полисистема~--- также система, а полисистемный анализ~--- новая форма системного анализа.

Процедура расслоения до очевидного проста и напоминает сортировку предметов по видам. Математика придала ей строгую и законченную форму.

\section{Основные понятия}

Расслоением называется \cite{b335} непрерывное отображение $\pi{}$ пространства $X$ на пространство $B$\,:
\begin{equation}
\pi{}:X\to{}Y\qquad{}\text{или}\qquad{}X\overset{\small{}\pi}{\to} B. \label{f:1:1}
\end{equation}

Расслоением является как сам процесс отображения $\pi{}$, так и вся триада $\xi{} = (X, \pi{}, B)$. Пространство $X$ называется пространством (объектом) расслоения, $B$~--- базой расслоения, $\pi{}$~--- проекцией расслоения. Обратное отображение $f = \pi^{-1}$ такое, что $f: B\to{} X$ называется сечением расслоения и представляет пространство $X$ в виде расслоенного множества (расслоенного пространства), т.~е. множества непересекающихся подмножеств. Для любой точки $b\in{}B$ ее прообраз $Y_b = f(b)$ называется слоем расслоения $\pi{}$ над точкой $b$. Расслоение с базой $B$ называется также расслоением над $B$.

Например, расслоить растительное сообщество по флористическому составу~--- значит рассортировать все особи по их принадлежности к тому или иному виду. В данном случае объект расслоения $X$~--- растительность, отображение $\pi{}$~--- сопоставление отдельных особей сообщества с конкретными видами (элементами $b$) из определителя растений (базы расслоения $B$), обратное отображение $f$~--- выделение растений в самом сообществе как представителей этих видов. Слой (моносистема) $Y$ соответствует совокупности (множеству) всех растений одного вида в сообществе. Полисистема $\{Y_b\}$~--- множество слоев, т.~е. все растения, рассортированные по видовой принадлежности. Аналогично проводится сортировка химических элементов вещества.

В процедуре расслоения существуют два представления пространства $X$\,: пространство"=объект расслоения $X$ и расслоенное пространство $Y = \{Y_b\}$, состоящее из слоев $Y_b = f(B)$. Они связаны коммутативной диаграммой

(2)\label{f:1:2}

Коммутативные диаграммы являются основным средством представления знаний в математической теории категорий и полисистемном анализе. В диаграммах стрелками обозначаются отображения множеств (морфизмы). Причем выполняется коммутативный принцип\,: суперпозиция (наложение) отображений по разным маршрутам эквивалентна. Например, в диаграмме (\ref{f:1:2})\footnote{Ссылки на формулы внутри раздела главы даются в виде одной цифры в скобке, например (3), из другого раздела в виде двух~--- (5.3), из другой главы в виде трех~--- (1.5.3).} последовательное применение (суперпозиция) двух процедур $\pi{}$ и $f$ в точности должна быть равна прямому отображению $\gamma = \pi{} \circ{} f$.

Расслоение физико"=географического объекта $X$ (ландшафта, провинции) по типологическому критерию соответствует процедуре типологического картографирования, когда каждый участок местности включается в тот или иной типологический ареал в зависимости от того, какому типу $b$ геосистемы он соответствует. Здесь слоем (моносистемой) $Y_b$ является совокупность всех ареалов геосистем одного типа $b$ (геомера) в границах выделенного географического пространства. Геохора (полисистема по Преображенскому)~--- совокупность разнотипных ареалов $\{Y_b\}$. Здесь базе расслоения $B$ соответствует полный набор классификационных типов геосистем. Если в базе расслоения отсутствует хотя бы один элемент, картографирование завершить невозможно. В этом выражается требование минимальности $B$\,: изъятие из базы $B$ хотя бы одного элемента делает операцию полного расслоения объекта невозможной.

Исторически первый удачный пример неявного использования методологии полисистемного анализа~--- изобретение алфавита. До него каждое слово записывалось иероглифами. Великое открытие сделали те, кто понял, что произнесенное слово членораздельной речи может быть разбито (расслоено) на элементарные части"=слоги (морфемы) и отдельные звуки (фонемы). Как бы слова ни отличались друг от друга, но выявляется база расслоения (алфавит), выраженная в символах~--- буквах. Второй удачный опыт применения идеи расслоения~--- периодическая таблица элементов Менделеева.  Идея расслоения всех веществ на химически элементарные составляющие объяснила многое в теории вещества и дала толчок развитию современных физики и химии.

Простой и наглядный пример расслоения дает разбиение прямой \cite{b136}. Ориентированная прямая х просто расчленяется точками $x$ на множество отрезков $(x_b, x_{b+1})$ разной длины\,:

%picture

\noindent{}Здесь пространство расслоения $X$~--- все точки прямой $x$; базис расслоения $B$ задается множеством индексов"=номеров $b$; расслоенное пространство состоит из множества отрезков $Y_b = [x_b, x_{b+1})$; проекция расслоения $\pi{}$~--- некоторое правило, связывающее точки $x$ с индексами $b$. Очевидно, что отображение $\pi{}: x\to{} b$ имеет место, если $x_b \leqslant{} x < x_{b+1}$. На изолинейных картах область $x_b \leqslant{} x < x_{b+1}$, соответствует пространству, заключенному между двумя изолиниями $x$ и $x_{b+1}$. Эта часть пространства соответствует в данном случае слою, или моносистеме. Полисистема~--- совокупность такого рода подпространств. Примерно так строится любая шкала картографируемых показателей, например шкала высот. Однако при введении шкал основное внимание уделяется заданию границ $x_b$ интервалов и упускается из виду, что за этим разбиением всегда стоит множество элементов $b\in{}B$ базы расслоения, которое в итоге и задает порядок и свойства этих интервалов. В частности, величины $x_b$ можно рассматривать как функции $b$, например для равномерной шкалы высот $x_b=\Delta{}x\cdot{} b$, где $\Delta{}x$~--- шаг деления оси $x$; $b$~--- в данном случае числа $b=0,$ $\pm{}1,$ $\pm{}2\ldots{}$ В итоге получается, что каждому отрезку можно сопоставить точку и число. Поэтому любая последовательность чисел натурального ряда $1,$ $2,\ldots{}$ задает базу расслоения $B$.


% fig 3, fig 4

Другой пример~--- расслоение двумерного пространства"=плоскости (рис.~\ref{fig:1:3}). База расслоения $B$ соответствует точкам линии $B$ с номерами от 1 до 7. Слои задаются параллельными линиями как множество точек, проходящих только через одну точку из $B$ и непересекающихся друг с другом. Каждая линия~--- моносистема, параллельный пучок линий~--- полисистема. Линии могут быть и непараллельны, например проходить через точки $B$ и через одну общую точку $\beta$. Если $\beta{}$ не рассматривать как элемент пространства расслоения, то такой центральный пучок (см. рис. \ref{fig:1:1}) состоит также из непересекающихся множеств~--- прямых линий. Поскольку в данном случае точки $B$ линейно упорядочены, то $B$ определяет некоторый порядок в пространстве расслоения, что является одной из главных причин обращения к методам полисистемного анализа.

Подобный подход можно перенести на многомерное пространство $Z=\{z_l\}$, задавая слои линиями координат количественных ха-

%page 26 follows

\end{document}
