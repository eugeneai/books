\documentclass[12pt,draft,openany]{extbook}
\usepackage[lit,fancytop,ptfonts]{subook}
%\usepackage[lit,fancytop,ptfonts]{subook}
\setlength{\parindent}{3.5ex}
\begin{document}
\begingroup
\thispagestyle{empty}
\mbox{}

\ptsans
\vspace{13ex}
\centering\large
\noindent{}A.~K.~Cherkashin

\vspace{6ex}

{\Huge\bfseries

POLYSYTEM

ANALYSIS

AND SYNTHESIS

}
\vspace{1em}

The application in geography
\\[0.7em]
{\normalsize{} Editor {\itshape{} V.~S.~Mikheev}}
\vfill{}

Irkutsk 1997
\mbox{}
\endgroup
\newpage
\begingroup
\mbox{}
\thispagestyle{empty}

\ptsans
\vspace{13ex}
\centering\large
\noindent{}А.~К.~Черкашин

\vspace{6ex}

{\Huge\bfseries

ПОЛИСИСТЕМНЫЙ

АНАЛИЗ

И СИНТЕЗ

}
\vspace{1em}

Приложение в географии
\vspace{1em}

{\normalsize\parskip0pt
Ответственный редактор\\[-0.5em] доктор географических наук {\itshape В.~С.~Михеев}}
\vfill{}

Иркутск 1997
\mbox{}
\endgroup
\newpage{}
\begingroup
\newcommand\sucopyright{{\ptserif\copyright}}
\newcommand\ISBN{ISBN 5--02--030607--X}
\thispagestyle{empty}
\noindent{}УДК 167:510:910

\noindent{}ББК 22.172

\vspace{2em}

\noindent{}\begin{minipage}[t]{3em}
\noindent{}Ч 48%
\end{minipage}\hfill{}%
\begin{minipage}[t]{0.92\linewidth}
\setlength{\parindent}{3.5ex}

Черкашин~А.~К. {\bfseries Полисистемный анализ и синтез. Приложение в географии}\,{}/
А.~К.~Черкашин.~-- Новосибирск: Наука. Сиб. предприятие
РАН, 1997.~--~\pageref{LastPage}~с.\\[0.3em]
\ISBN{}.
\vspace{1em}

\begin{abstract}\
В книге исследуется сложный объект (планета Земля и свойственные ей процессы и явления) с разных сторон, в различных системных проекциях с использованием полисистемной методологии, возникающей из логики анализа противоположностей, понятий и законов общей теории систем в авторской интерпретации и из современных математических представлений о многомерных расслоенных пространствах. Для каждой научной проекции разрабатывается аксиоматическая теория, описывающая на специальном системном языке сквозным образом природные, экономические и социальные структуры и их изменения. Выделено около 20 теоретических направлений, отражающих различные аспекты изучения географических явлений разного масштаба, упорядочивающих имеющееся знание и позволяющих получить новые объяснения фактам. Для каждого направления дается краткий анализ понятийной базы, излагается используемый математический аппарат и на примерах иллюстрируются полученные выводы.

Книга представляет интерес для исследователей, работающих к различных областях науки и интересующихся проблемами поиска оснований формирования теоретических знаний и развития системной методологии решения конкретных задач.

Табл. 8. Ил. 94. Библиогр.: 563 назв.
\end{abstract}

\vspace{2em}
\centering
\textbf{Рецензенты}

доктор географических наук \textit{Ю.~М.~Семенов}

доктор физико-математических наук \textit{Г.~Н.~Константинов}
\vspace{1ex}

Работа выполнена при финансовой поддержке

Российского фонда фундаментальных исследований.

Проект 93-05-14076

"Теория и методы полигеосистемного анализа".

Проект 96-05-64727

"Геоинформационные процессы: исследование и моделирование"
\end{minipage}

\vfill{}
\begin{minipage}[t]{20em}
\mbox{}\par
\ISBN{}
\end{minipage}\hfill
\begin{minipage}[t]{0.7\linewidth{}}
\noindent\sucopyright~А.~К.~Черкашин,~1997\par
\noindent\sucopyright~Российская академия наук, 1997
\end{minipage}
\endgroup
\newpage
\tableofcontents{}
\chapter*{Предисловие}
Логической базой всей географии была выработанная к концу XIX в. (А. Гумбольдт, В. Докучаев и др.) особенность изучения целостной картины природы, с помощью которой для осмысления естественных отношений использованы цельные зрительные образы неопределенной формы, в точности соответствующие пониманию природных объектов как "многого в едином". Поскольку эти образы были зрительными, они картографировались, поскольку находили отражение в конкретных явлениях, то выражались через природные компоненты и их связи, что способствовало развитию отраслевых разделов географии. На теоретические основания классического ландшафтоведения (начало~--- середина ХХ в.) были возложены надежды на установление и описание всеобщей связи явлений и их объективной субординации. Началась онтологизация понятий, обеспечивающих целостность восприятия, что содействовало быстрому становлению теоретических основ современной географии, в частности аппарата исследований для получения новых фактов и проверки истинности научных выводов, что привело к конкретным результатам, по-новому характеризующим природные и социально-экономические образования, например в учении о зональности. Тем самым был создан необходимый фундамент для перевода многих представлений географии из разряда интуитивных в разряд более точных.

В последние годы по ряду объективных причин география решительно перешла от визуальных наблюдений и обобщений к научно организованным исследованиям в контролируемых условиях. Она берет на вооружение и разрабатывает методические подходы, основанные на применении современных приборов и новых технологий для решения комплексных географических проблем. Необходимо понять, как вовлекаются в действие различные структуры географических объектов, как их анализировать: по отдельности, с последующим изучением их взаимосвязи или сразу, взятые в целом? На этом пути необходимо, преодолевая инерцию внутригеографического мышления, найти общенаучный подход к теории и практике операций с комплексными объектами. Формируется тенденция, предъявляющая возросшие требования к метатеории географии и выявившая несоответствия между имеющимися гносеологическими принципами и новыми научными фактами, что привело если не к дискредитации наивной теоретико-познавательной основы классической географии, то к известной неудовлетворенности ее постулатами и пониманию необходимости их замены более строгими положениями.

Монография доктора географических наук А.~К.~Черкашина в этом смысле имеет особый характер. Это оригинальный труд ученого, который, обладая большим личным опытом исследовательской работы в различных областях знания, стремится сформулировать новые задачи, возникшие перед географической наукой, и наметить пути и методы их решения. В центре внимания автора~--- теория и методология полисистемного анализа и синтеза, значение которых читатель начинает понимать только тогда, когда он воспринимает явления и объекты географической реальности как сложные образования, которые невозможно описать в рамках одной теории. При этом полисистемный анализ выступает как специальное направление конструктивного исследования сложных объектов. Его методы исходят из теоретического и методического базиса общей теории систем, формальной и диалектической логик, успешно применяемых разными науками. Идея полисистемности строения природы, а вместе с ней созданных человеком социально-экономических образований, на фоне недостаточности информации о их свойствах, процессах и состояниях вносит новые элементы в системные исследования географической среды. Во-первых, возникает необходимость в ином подходе к классификации системных качеств, методов их описания, выделения и закрепления в научной практике. Во-вторых, что принципиально, в системных исследованиях при реализации принципа объективности выделения систем разного рода возникает информационный барьер между качественной и количественной соизмеримостью неоднозначных, неравноценных и недостаточных фактов, поэтому содержательное изучение определенного типа систем необходимо дополнять совокупностью исследовательских средств и методов, обеспечивающих достоверность и упорядоченность информации в системе знаний. В-третьих, требуется новый подход к созданию информационной базы изучения природного целого и ее формализации для описания зависимостей между элементами и связями системы.

Способы получения и использования знаний разрабатываются на разных уровнях. В географии созданы и продолжают развиваться исследования, основанные на методах сбора информации с последующим эмпирическим обобщением данных. Естественно, они никогда не достигнут стадии полного завершения, так как любой опыт не может быть интерпретирован в абстракции от понятийного и логического аппарата, который делает возможным научную интерпретацию. Сам факт чувственного восприятия предполагает активность нашего сознания и является продуктом не только одних "чувственных сигналов", как например у Э.~Маха, но и размышления, принимающего в расчет многоступенчатый критерий практики и учитывающий не только факты реальности в "наивном", несистематизированном и логически не оформленном понимании, но и их связь.  Сознание проделывает большую селекционную и классификационную работу, что приводит к необходимости находить новые, обновлять традиционные или приспосабливать заимствованные из разных наук методические приемы, процедуры и исследовательские операции. Именно систематизации методов, определенной перестройке оснований географии и ее постепенной строгой формализации, как это было в других точных науках, например в физике и математике, явившихся ориентиром создания новых познавательных средств, и посвящено исследование А.~К.~Черкашина.

Им подняты важные вопросы обобщения опыта получения знаний на этапе перехода от эмпирического к теоретическому уровню с использованием процедур и алгоритмов разных видов исследовательской работы. Речь идет не просто о необходимости выделения и проведения познавательных операций над наблюдаемыми процессами и явлениями природы, а операций над комплексными объектами, которые не могут быть объяснены и предсказаны, опираясь только на "здравый смысл", они требуют создания сложных и многоступенчатых понятийных теорий. Суть заключается в том, что научный географический анализ на первых порах был "наукой о природе", где нерасторжимо слиты естественно-научные факты, выработанные человеческим интеллектом,~--- интуицию и логика, проникшие из естественного языка самой природы в описательную географию. Изучение логических аспектов анализа не было актуальным. Теперь положение изменилось, поскольку, имея дело с миром сложных систем, географам стало трудно сверять такую теорию "непосредственно с вещами материального мира" и требуется своеобразный компас, позволяющий не сбиться с пути и вести исследования правильным способом. На роль такого компаса все больше начинает претендовать строгий логический метод рассуждения, в результате чего потребность в изучении и применении законов логики значительно возросла. Полисистемный подход уделяет особое внимание моделированию сложных объектов природы и использованию логических и математических методов. Этой важной стороне объективного развития географической науки в работе А.~К.~Черкашина отведено значительное место.

В поле зрения автора находится проблема взаимодействия и коммуникации различных областей знания, чем, в частности, объясняется большое внимание к моделям и понятийно-терминологической базе исследований. Такой подход применен для формирования системы понятий и показа широких возможностей оперирования элементами языка теории~--- аксиомами, терминами, определениями~--- по четким правилам логики с обоснованием достоверности полученных результатов. Автор обобщил методы формализации многих наук, поставил их знания "на службу географии", чем выразил давнее стремление географов осознанно отражать факты в понятиях более широкого охвата при описании и выделении разных природных и социально-экономических образований. По существу, им представлена миогоаспектная программа формализации географических законов и закономерностей для выделения генеральных сквозных направлений теоретического обобщения результатов исследований и особенностей методических действий при решении конкретно-научных проблем, а также показаны возможности их применения в географии и других науках. Посредством таких представлений техника моделирования и математические модели естественным образом соединяются с содержательной географией эмпирических обобщений.

Конечно, следует отдавать себе отчет в том, что научная проблематика, о которой идет речь в книге, очень сложна, многогранна и еще недостаточно разработана, поэтому поворот современной географической науки в эту сторону еще далеко не завершен и дает, как это видно из исследования А.К. Черкашина, первые полезные результаты. Рассмотрение им проблемы содержат элементы неопределенности и дискуссионности. Познание комплексных природных и других объектов как сложных полисистем~--- это процесс более трудный и миогоступенчатый, чем было принято думать раньше. Он связан с построением миожества вспомогательных научных понятий, не имеющих прообразов в реальной природе и познавательных ситуациях, но дающих возможность перейти от общих философских категорий к конкретным уравнениям и расчетам, показав логические истоки возникновения каждой закономерности.

Исследования А.К. Черкашина позволяют развить экстенсивное представление о структуре географического исследования через обоснование существования уровней и типов исходной информации, получаемой из различных. источников, которые, будучи обобщенными едиными методами, сводятся в итоге к эквивалентным интерпретациям научных знаний. Это восполняет пробел между традиционным изучением географических явлений и географией, идущей ему на смену.

\begin{flushright}
\small{}
Доктор географических наук\par{}
\itshape{}
В.С. Михеев
\end{flushright}
\end{document}
