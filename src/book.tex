
\documentclass[12pt,leqno]{book}
\usepackage{amsmath,amssymb,amsfonts} % Typical maths resource packages
\usepackage{graphics}                 % Packages to allow inclusion of graphics
\usepackage{color}                    % For creating coloured text and background
\usepackage{hyperref}                 % For creating hyperlinks in cross references
\usepackage{indentfirst}

\parindent 1cm
\parskip 0.2cm
\topmargin 0.2cm
\oddsidemargin 1cm
\evensidemargin 0.5cm
\textwidth 15cm
\textheight 21cm
\newtheorem{theorem}{Theorem}[section]
\newtheorem{proposition}[theorem]{Proposition}
\newtheorem{corollary}[theorem]{Corollary}
\newtheorem{lemma}[theorem]{Lemma}
\newtheorem{remark}[theorem]{Remark}
\newtheorem{definition}[theorem]{Definition}


\def\R{\mathbb{ R}}
\def\S{\mathbb{ S}}
\def\I{\mathbb{ I}}
\makeindex


\title{Polysystem Analysis and Synthesis\\\textsc{\small{} The application in geography}  }

\author{A.K.Cherkashin  \\
{\small\em Editor in chief of the Russian version V.S.~Mikheev }}

 \date{ }
\begin{document}
\maketitle
 \addcontentsline{toc}{chapter}{Contents}
\pagenumbering{roman}
\tableofcontents
\listoffigures
\listoftables
\newpage{}
\pagestyle{empty}
\noindent{}UDC~167:510:910\\{}
BBK 22.172 (Russian version)
\vfill
\begin{quote}
\small{}
\textbf{Polysystem analysis and synthesis. The application in geography.} /\\{}
A.K.~Cherkashin. Russian version of the book published by Nauka Publishing Enterprise at Siberian Branch of Russian Academy of Sciences, Novosibirsk. 1997.

ISBN 5-02-030607-X (in Russian)

The subject of the book is to develop a general methodology of a complex object (the Earth and peculiar to it phenomena and processes) exploration from various aspects (view points), in various system projections using the polysystem methodology, that arise from the contradiction analysis logic, the notions and the laws of General Theory of Systems in an author's interpretation, as well as from nowadays mathematical paradigm of multidimension fibered spaces [bundles]. For each scientific view [projection] an axiomatic theory is being developed. The theory describes in a special system language [and in a through way] natural, economic, social structures and their changes. About 20 theoretical directions were highlighted, which reflect various aspects of geographical phenomena investigation of various scales, structuring existing knowledge and allowing to obtain new explanation to [the] facts. A short analysis of the conceptual basis is presented for the directions, used mathematical formalism and tools are introduces, the arrived conclusions are illustrates by examples.

The book could be of interest to the researchers of various scientific fields, which are interested in exploration of theoretical knowledge formation foundations and development of a system methodology for concrete problem solving [investigation].

\vspace{1em}

\begin{center}
\textsc{Reviewers of the Russian version}\\{}
Yu.M.~Semenov, Professor of Geography\\{}
G.N.~Konstantinov, Professor of Mathematics\\{}
\vspace{1em}
The Russian version of the book was published with support of\\{} Russian Foundation of Fundamental Research,\\{}
Grant No.~93-05-14076\\{}
``The theory and methods of polysystem analysis''.\\{}
Grant No~96-05-64727\\{}
``Geoinformation processes: research and modeling''.
\end{center}
\vfill{}
\begin{tabular}{lcl}
{}\hspace{0.4\linewidth}{} & & \copyright{} A.K.~Cherkashin, 1997\\
& & \copyright{} Russian Academy of Sciences, 1997\\
& & \copyright{} E.A.~Cherkashin (translator), 2014
\end{tabular}
\end{quote}
\chapter*{Preface}\normalsize
  \addcontentsline{toc}{chapter}{Preface}
\pagestyle{plain}
The book root file {\tt bookex.tex} gives a basic example of how to
use \LaTeX \ for preparation of a book. Note that all
\LaTeX \ commands begin with a
backslash.

Each
Chapter, Appendix and the Index is made as a {\tt *.tex} file and is
called in by the {\tt include} command---thus {\tt ch1.tex} is
the name here of the file containing Chapter~1. The inclusion of any
particular file can be suppressed by prefixing the line by a
percent sign.


 Do not put an {\tt end{document}} command at the end of chapter files;
just one such command is needed at the end of the book.

Note the tag used to make an index entry. You may need to consult Lamport's
book~\cite{lamport} for details of the procedure to make the index input
file; \LaTeX \ will create a pre-index by listing all the tagged
items in the file {\tt bookex.idx} then you edit this into
a {\tt theindex} environment, as {\tt index.tex}.


\pagestyle{headings}
\pagenumbering{arabic}

\include{ch1}
\include{ch2}

\begin{thebibliography}{99}
  \addcontentsline{toc}{chapter}{Bibliography}
\bibitem{lamport} L. Lamport. {\bf \LaTeX \ A Document Preparation System}
Addison-Wesley, California 1986.
\end{thebibliography}

\include{index}
  \addcontentsline{toc}{chapter}{Index}
\end{document}
