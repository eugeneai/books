
\documentclass[12pt,leqno]{book}
\usepackage{amsmath,amssymb,amsfonts} % Typical maths resource packages
\usepackage{graphics}                 % Packages to allow inclusion of graphics
\usepackage{color}                    % For creating coloured text and background
\usepackage{hyperref}                 % For creating hyperlinks in cross references
\usepackage{indentfirst}

\parindent 1cm
\parskip 0.2cm
\topmargin 0.2cm
\oddsidemargin 1cm
\evensidemargin 0.5cm
\textwidth 15cm
\textheight 21cm
\newtheorem{theorem}{Theorem}[section]
\newtheorem{proposition}[theorem]{Proposition}
\newtheorem{corollary}[theorem]{Corollary}
\newtheorem{lemma}[theorem]{Lemma}
\newtheorem{remark}[theorem]{Remark}
\newtheorem{definition}[theorem]{Definition}


\def\R{\mathbb{ R}}
\def\S{\mathbb{ S}}
\def\I{\mathbb{ I}}
\makeindex


\title{Polysystem Analysis and Synthesis\\\textsc{\small{} The application in geography}  }

\author{A.K.Cherkashin  \\
{\small\em Editor in chief of the Russian version V.S.~Mikheev }}

 \date{ }
\begin{document}
\maketitle
\pagenumbering{roman}
\newpage{}
\pagestyle{empty}
\noindent{}UDC~167:510:910\\{}
BBK 22.172 (Russian version)
\vfill
\begin{quote}
\small{}
\textbf{Polysystem analysis and synthesis. The application in geography.} /\\{}
A.K.~Cherkashin. Russian version of the book published by Nauka Publishing Enterprise at Siberian Branch of Russian Academy of Sciences, Novosibirsk. 1997.

ISBN 5-02-030607-X (in Russian)

The subject of the book is to develop a general methodology of a complex object (the Earth and peculiar to it phenomena and processes) exploration from various aspects (view points), in various system projections using the polysystem methodology, that arise from the logic of contradiction analysis, the notions and the laws of general systems theory in an author's interpretation, as well as from nowadays mathematical paradigm of multidimension fibered spaces [bundles]. For each scientific view [projection] an axiomatic theory is being developed. The theory describes natural, economic, social structures and their changes with special system language [and in a through way]. About 20 theoretical directions were highlighted, which reflect various aspects of geographical phenomena investigation of various scales, structuring existing knowledge and allowing to obtain new explanation to [the] facts. A short analysis of the conceptual basis is presented for the directions; used mathematical formalism and tools are introduced; the arrived conclusions are illustrates by examples.

The book could be of interest to the researchers of various scientific fields, which deal with studying of theoretical knowledge formation foundations and development of a system methodology for concrete problem solving [investigation].

\vspace{1em}

\begin{center}
\textsc{Reviewers of the Russian version}\\{}
Yu.M.~Semenov, Professor of Geography\\{}
G.N.~Konstantinov, Professor of Mathematics\\{}
\vspace{1em}
The Russian version of the book was published with support of\\{} Russian Foundation of Fundamental Research,\\{}
Grant No.~93-05-14076\\{}
``The theory and methods of polysystem analysis''.\\{}
Grant No~96-05-64727\\{}
``Geoinformation processes: research and modeling''.
\end{center}
\vfill{}
\begin{tabular}{lcl}
{}\hspace{0.4\linewidth}{} & & \copyright{} A.K.~Cherkashin, 1997\\
& & \copyright{} Russian Academy of Sciences, 1997\\
& & \copyright{} E.A.~Cherkashin (translator), 2014
\end{tabular}
\end{quote}

\tableofcontents

\chapter*{Preface of editor in chief}\normalsize
  \addcontentsline{toc}{chapter}{Preface of editor in chief}
\pagestyle{plain}



A peculiarity of the studies of the holistic picture of the nature had been elaborated as the logical foundation by XIX-th century (A.~von~Humboldt, V.~Dokuchaev and others): a whole visual identity of an uncertain [?unsystem?] form [corresponded] to a [representation] of natural objects as ``many--in--one'' had been used for interpretation the nature patterns [relation, relationships]. As the identities was visual, they were mapped; as they corresponded to concrete phenomena, they were interpreted [via] natural components and causalities. This contributed to sectoral branches of geography. Expectations to determination and description of the universal relationship of the phenomena and their object subordination were entrusted to the theoretical basis if the classical landscape science ([beginning--middle] of XX--th century). The notions ontologisation started providing the integrity of perception, resulting in prompt establishment [formation] of the nowadays geography, in particular, the research method for new facts acquisition and scientific conclusions verification. This gave the concrete results characterizing natural and social--economic formations in a new fashion, for example, in the zonal concept. The advance realized the necessary foundation to shift major geographical interpretations from the intuitive ranks to the rank of more precise ones.

In recent years, thanks to a number of objective reasons geography promptly switched from visual observations and generalizations to scientifically organized research in controllable environments. It adopts and develops methodological approaches based on contemporary instruments and new technologies for solving complex geographical problems. Various structures of geographical objects are involved in effects. The essence of the involvements is to be understood. How they should be analyzed? --- As different independent structures followed by their causalities studies, or as whole straight away. Overcoming inertia of intrageographical way of reasoning, a general scientific approach to theory and practice of the complex objects operations have to be developed. A trend is being formed that imposes heavy demands to geography metatheory. It recognized disparity [contradictions] epistemological principles and new facts, that almost resulted in discrediting the naive epistemological foundation of classical geography. The known discontentment to its postulates and the necessity perception of their replacement are the outcome of [the trend] as well.

The monograph of Professor A.K.~Cherkashin in this regard has a distinctive character [nature?-]. This is an original effort of a scientist who on the base of the great personal experience of the research work in different knowledge areas seeks to formulate new problems emerging to the geographical science, and to trace ways and methods of their solutions. The center of author's attention is the theory and the methodology if polysystem analysis and synthesis. Their importance will be understood [by reader] when the objects of the geographical matter [reality] as complex formation, which cannot be described [interpreted] within one theory, has been accepted, with the polysystem analysis being the special field of the [a] constructive research of [the] complex objects. Its methods originated from theoretical and methodological basis of the general systems theory, the formal and dialectical logics, which are successfully applied in various sciences. The idea of polysystem organization of nature together with social formations created by human being on the background of information lack on their properties, processes and states introduces new elements in the system research of the geographical environment. Firstly, the necessity appears to elaborate [create] a new [diverse] way of systemic qualities classification, methods of their description, recognition and fixation in the science practice. Secondly, and it is essential, in the system research, realizing the objectivity concept in recognition of the systems of various kinds, the informational barrier appears between qualitative and quantitative measurement [commensurability] of ambiguous, unequal and insufficient facts. That's why the informative study of the systems of a certain class must be completed with a set of the research [tools, ware] and methods supplying validity and regularity of the information in a knowledge system. Thirdly, a new approach of creation of an informational basis [base] of the [nature integrity] study and its formalization for description of the elements dependence and the system's outer relationships must be developed.

The ways of the knowledge acquisition and utilization [usage] are developed at the various levels [...of...]. In geography, the studies based on information [collection] techniques and the followed empirical data generalization are developed and still in an active development. Naturally, they will not came up to the total completion as the [any] experience should not [cannot] be interpreted in the abstract[ion] to [a] conceptual and logical basis, which enables the scientific interpretation. The fact of the sensory perception itself implies the activity of our consciousness, and it is the result not just ``sensory signals'' by (E.~Mah??), but it involves the reasoning, which takes into account the multistage criterion of the [practice, experience] aggregating the facts of the reality not only as ``naive'' unsystematic and logically unformal interpretation, but as their dependencies. The consciousness performs a great selection and classification work, that results in necessity to find new, renew the traditional or adopt taken from various sciences methodological ways, procedures and research operations. This investigation by A.K.~Cherkashin is devoted to a methods systematization, definitive restructuring geography foundation and its sequential explicit formalization, as it had been [was] in other precise sciences, for example, in physics and mathematics, which are the reference point of new cognitively means.

He recognized important problems of knowledge acquisition experience generalization, using procedures and algorithms of various diversity of the research study, on the stage of transition from the empirical to the theoretical level. It is not just a question of necessity of separation [individualization] and application of the epidemiological operations to observable processes and natural phenomena, but operations over complex objects, which cannot be interpreted and predicted on the base of ``common sense'', they [?objects?] require elaboration of complex multistage conceptual theories. The essence is that scientific geographical analysis in the beginning was ``the science on nature'', where natural science facts delivered by human intelligence are fused inseparably, --- intuition and logic [entered, infiltrated] from language of nature itself to descriptive geography. [The] Study of the logical aspects of [the] analysis was not actual. Now the conditions has changed, having to deal with the world of complex systems, it became difficult for geographers to match [the, such] theory [of such kind] with ``the things of the material world'' [themselves], and a special compass have to be constructed disallowing to be gone astray and directing the research along the right path. The strict logical method of reasoning staking a claim to the role of such compass. That's why need for studying and application of the laws of logic increased substantially. The polysystem approach pays particular attention to modeling complex objects of nature and the usage of logical and mathematical methods. This important aspect of the objective development of geographical science is the main subject of the book [of A.K.~Cherkashin].

The problem of interaction and communications of various fields of knowledge is in the author's field of view. In particular, this is due to his great attention to the models and the conceptual base of the studies [research]. This approach has been applied to form a system of notions and to demonstrate manifold capabilities of manipulation of language elements of a theory --- axioms, notions, definitions --- according to precise rules of logic with argumentation of the validity of the obtained results. Author has generalized the formalization methods of many sciences, their knowledge has been placed in the service to geography. This expresses the longstanding desire of geographers to consciously reflect the facts in notions of a wide coverage in recognizing and describing natural and socio--economic formations. In essence, he elaborated a multiaspect scheme [blueprint] of formalization of the geographical laws and patterns for recognition of the global [general] through directions of theoretical generalization of the study [research] results [and features of methodological operations in the concrete-science problems solving]. Their application capabilities are demonstrated in geography and other sciences [as well]. By means of the representations modeling techniques and mathematical models conjugate to the meaningful geography of empirical generalizations in a natural way.

Of cause, one must realize that scientific problematics the book devoted to is complex, versatile and not well elaborated. Therefore, the transition of the present geography science to this direction is far to be finished, but it gives [presents] as it can be seen from A.K.~Cherkashin's research the first useful results. The considered problems contain elements of doubtfulness [uncertainty] and discussion [question]. Understanding complex natural and other objects as a complex polysystems is rather difficult and multistage process [than thought before]. It involves the construction of many scientific concepts that have no prototypes in the real nature and cognitive situations. The concepts allow one to move from general philosophical categories to the concrete equations and evaluations, expressing logical origins of each pattern emergence [occurrence]. The research of A.K.~Cherkashin allows to develop ``extensive'' interpretation of the structure of geography study via foundation of existence of levels and types of the initial information, acquired from various sources, which having been generalized with [unified] method, are reduced in the issue to equivalent interpretations of the scientific knowledge. This fills the gap between traditional investigation of the geographical phenomena and geography itself, coming to replace it.

\vspace{1em}
\noindent{} \hfill{} Professor V.S.~Mikheev, Geography Sciences

\chapter*{Introduction}\normalsize
  \addcontentsline{toc}{chapter}{Introduction}

The main challenge [problem] of the contemporary science is the problem of comprehension of complex [objects]. Experience provides plenty of evidence to ensure that any complex phenomenon will not become simple[r] whenever extending our knowledge, instead, it converts to something more complex and incomprehensible. Therefore, the complexity as an universal quality of objects should be discussed not with respect to our knowledge level, but it should be regarded with the same degree of reality as other fundamental properties of objects. This approach requires the researchers to use the corresponding methodology and the logic of thinking to have new vision of the established notions on processed and phenomena.

In science, there are fields of knowledge that deals with complex objects, phenomena and processes, and it is the existence basis of their subject of investigation. They are ancient so called synthetic domains, which include in the first place history, medicine and geography. Known Latin adage ``\emph{historia est mater studiorum}'' reflects the essence of the historical cognition as the starting point of any human knowledge.

Some reasons hider the holistic synthetic cognition of the complex processes and phenomena. Firstly, it is not clear what is to be synthesized. It seems that the scientific knowledge differentiation did not reach yet the necessary degree of desalination to recognize the bricks of scientific knowledge, which form the picture of the world from. Secondly, the principles of system synthesis did not well established yet. Thirdly, the logics and the methodology, which are necessary for the analysis and synthesis, are not sufficiently completed.

Historically the logic of the present science was founded more than two thousands year ago in the works of Aristotle on the formal logics. It is the formal logics become the ideal of the scientific thinking and primarily in mathematics and physics, which resulted to a great extent in their permanent progress. Nowadays computerization is [almost, the] explicit expression of the triumph of the formal logics as computers function [almost] exclusively on the base of the principles and the properties of binary calculi peculiar to it.

% page 16
The logic of life [nature] and human thinking, however, goes far beyond the formal binary algorithms, it is demonstrated especially well by expert systems [software] development. The meaningful knowledge of the experts are realized as computer programs with difficulty, and the obtained software [usually] is inferior as compared to the capabilities of the experts.

The understanding the limitations of the formal logic gave raise the invention of a number of new logical systems, among which the dialectical logic is distinguished. It is obscure and in the same time the nearest to the reality logical system. The dialectic logic [is] known to be formed in the Hegel's works, but it got no distribution as an instrument of applied scientific analysis especially with axiomatic method that allows to infer knowledge from a compact set of the basic concepts and laws.

In order to understand the idea of a complex entity with theoretical means, one have to go by the paradoxical way --- through the deepest differentiation of the science and corresponding knowledge of the world. This way leads us to the elaboration of the concept of the polysystem analysis and synthesis. The essence of the polysystem analysis is the natural possibility to differentiate and stratify the big image of the complex observable objects on the system of non-intersecting sets of concepts and laws. The stratification lets us to imagine the projections of the natural objects and knowledge on them into each of similar set, representing the original whole entity in the system of coordinates of their particular mappings (layers [fibers, bundles]). The intended use of the polysystem analysis is to provide the procedure of fibering and to define on the set of the fibers (mappings) a structure and a system of fiber relations [!]. Usually the relations are defined as mutual mappings (comparison, equivalence, identity), various combinations of the fibers, morphisms of various kinds represented as commutative diagrams[-, -etc]. This is realized by means of the old [ancient] dialectical logic and nowadays procedures of abstract algebra.

Within this approach, the polysystem analysis affords new methods of systematization and knowledge conception [-representation]: after a proper structuring and deep fibering the individual knowledge fields, the corresponding axiomatic systems (theories) can be constructed in the image and likeness of the general systems theory or the dialectics. This is achieved via interpretation (a corresponding substitution) of the concepts of one theory with [!] concepts of other theory [one]. Utilizing this approach allows us to obtain surprising [unusual, amazing, novel, extraordinary] axiomatic systems for knowledge fields, which did not possess [have] them [? Knowledge fields or systems, Who owns what??] before.

%page 17
The solution of the problem has methodological value for all sciences. So, the famous physicist R.~Feinman [?] \cite{b437} wrote that ``our nowadays physical theories, laws of physics are the pile [set] of scattered [segmental] pieces and snippets badly harmonizing [matching] each other. Physics did not evolve into an united construction, where each part has its place'' [Probably find the original if any]. These words [This statement is] are still actual as well as many problems of the mathematical basis of D.~Hilbert stated 100 yeas ago \cite{b347}. Among them there is the sixth problem directly related to natural sciences: ``\ldots The investigations on the foundations of geometry suggest the problem: To treat in the same manner, by means of axioms, those physical sciences in which already today mathematics plays an important part;\ldots{}'' \cite[p.~34]{b347}. A half century later, L.~von~Bertalanffy, elaborating the general systems theory, also dreamed of the opportunity in its image and likeness to construct theories of other sciences (see \cite{b367}).

The ancient science [of] geography faced with the similar problems. It was a descriptive science upon appearance, and only from the middle of XIX century, it started to take shape as a relatively independent synthetic field of knowledge on the relationships and interaction of the objects and phenomena on the Earth surface \cite{b118,b176}. The further its differentiation lead to a situation similar to one happened in Physics and other sciences. Academition V.~B.~Sochava wrote with almost Feinmann words: ``Geography sciences, taken together, do not form yet a whole that would consist of organically tied [harmonized] parts or a set of interlinking elements arranged in a specific order'' \cite[p.~480]{b397}. To overcome this situation V.~B.~Sochava elaborated the study on the geosystems and specified the problematics and directions of the investigations. The first of them is problem of ``analysis of the axioms and other concepts of the special theory of geosystems as a parts of a general systems theory (metatheory)'' \cite[p.~15]{p397}.

The emphasis of theory of geography requires first of all recognizing the object and the subject of this science. Hereafter we will start from known ideas of A.~von~Humboldt that this science is a complex discipline. The bright representative of the complex approach in Earth sciences was V.~V.~Dokuchayev [may be write it vice versa]. In his program-project of detailed exploration of Saint Petersburg and its vicinity, a general principle was stated --- a principle of ``simultaneous complete comprehensive of a ``specific area [region]'', whose nature is to be studied as taken in a whole [,] unified and inseparable\ldots{}'' \cite[p.~461]{p152}.

The V.~V.~Dokuchayev's conception founded in the geography the idea of interconnection integrity between natural complexes, and it has been further developed in the landscape studies (L.~S.~Berg, B.~B.~Polynov), which appear to become in the future ``synthetic natural science'' and its ways of real world applications in the scope [area] of natural resources deployment [exploration]. The specific of the landscape science includes the study of the phenomena complex of Earth's outer shell. It cannot be reduced to the only domain of natural objects exploration [study] or to development of physical, chemical, biological and other laws \cite{b285}. With the formation of the landscape science theory (N.~A.~Solntsev, F.~N.~Milkov, A.~G.~Isachenk, D.~L.~Armand[t] et al), the general issues of methodology of the landscape studies received [a] development [were developed] \cite{b288}, this in various interpretations is expanded [developed, elaborated] in the works of many scientists. P.~Hagret [?] in his monograph \cite{b450}  emphasizes the synthesis [synthetic] role of the geography with respect to adjoining [neighbor, surrounding] fields of knowledge. We can talk about the formation, as it was figuratively mentioned by A.~A.~Krauklis \cite[p.~206]{b227}, a sort of ``geographical [-form of, -kind of] precision'' in the scientific research [+investigation,...], with the geographic entity [[al] object] is investigated every time as an unique phenomenon. The priority of the uniqueness in economical development of a territory and specialization of human activity becomes the most important prerequisite of planning and designing \cite{b351}.

%page 18
The notion of individuality and concreteness are well illustrated with concepts of consistent patterns [regularities] (complex situation) (fig.~\ref{fig:1}). Every event, phenomenon, process are a concatenation of circumstances, interference of various laws [crossing [of] various laws]. In this sense, the pattern [regularity] or a complex law is the intersection point of a bunch [bundle] of lines, each line corresponds to a law of interoperation of antithetical [contradictory] principles [-insceptions]. The [This,Such] model describes [defines] a general principle and a concept [symbol, figure, image] of geographic knowledge and way (!) of thinking.

\begin{figure}[tbhp] \label{fig:1}
\vspace{1em}
\caption{A geometric model illustrating the formation principles of patterns [regularities] or complex situations as an intersection [interference] of a variety of laws in the common space of geographical objects attributes ([according to] \cite{b24}). \protect\\\small{} The line $M\mbox{--}M^\prime{}$ in fig.~\emph{a}signify a certain law of nature representing two independent concepts $M$ and $M^\prime{}$; in figs.~\emph{b} and \emph{c}, two different patterns [regularities] being an intersection or interference of some specific laws ($P\mbox{--}P^\prime{}$, $Q\mbox{--}Q^\prime{}$, $N\mbox{--}N^\prime{}$, \emph{etc.}) are shown.  }
\end{figure}


%page 19




\pagestyle{headings}
\pagenumbering{arabic}

\include{ch1}
\include{ch2}

\begin{thebibliography}{99}
  \addcontentsline{toc}{chapter}{Bibliography}
\bibitem{lamport} L. Lamport. {\bf \LaTeX \ A Document Preparation System}
Addison-Wesley, California 1986.
\end{thebibliography}
\listoffigures
\listoftables
\include{index}
  \addcontentsline{toc}{chapter}{Index}
\end{document}
