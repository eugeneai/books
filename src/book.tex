
\documentclass[12pt,leqno]{book}
\usepackage{amsmath,amssymb,amsfonts} % Typical maths resource packages
\usepackage{graphics}                 % Packages to allow inclusion of graphics
\usepackage{color}                    % For creating coloured text and background
\usepackage{hyperref}                 % For creating hyperlinks in cross references
\usepackage{indentfirst}

\parindent 1cm
\parskip 0.2cm
\topmargin 0.2cm
\oddsidemargin 1cm
\evensidemargin 0.5cm
\textwidth 15cm
\textheight 21cm
\newtheorem{theorem}{Theorem}[section]
\newtheorem{proposition}[theorem]{Proposition}
\newtheorem{corollary}[theorem]{Corollary}
\newtheorem{lemma}[theorem]{Lemma}
\newtheorem{remark}[theorem]{Remark}
\newtheorem{definition}[theorem]{Definition}


\def\R{\mathbb{ R}}
\def\S{\mathbb{ S}}
\def\I{\mathbb{ I}}
\makeindex


\title{Polysystem Analysis and Synthesis\\\textsc{\small{} The application in geography}  }

\author{A.K.Cherkashin  \\
{\small\em Editor in chief of the Russian version V.S.~Mikheev }}

 \date{ }
\begin{document}
\maketitle
 \addcontentsline{toc}{chapter}{Contents}
\pagenumbering{roman}
\tableofcontents
\newpage{}
\pagestyle{empty}
\noindent{}UDC~167:510:910\\{}
BBK 22.172 (Russian version)
\vfill
\begin{quote}
\small{}
\textbf{Polysystem analysis and synthesis. The application in geography.} /\\{}
A.K.~Cherkashin. Russian version of the book published by Nauka Publishing Enterprise at Siberian Branch of Russian Academy of Sciences, Novosibirsk. 1997.

ISBN 5-02-030607-X (in Russian)

The subject of the book is to develop a general methodology of a complex object (the Earth and peculiar to it phenomena and processes) exploration from various aspects (view points), in various system projections using the polysystem methodology, that arise from the contradiction analysis logic, the notions and the laws of General Theory of Systems in an author's interpretation, as well as from nowadays mathematical paradigm of multidimension fibered spaces [bundles]. For each scientific view [projection] an axiomatic theory is being developed. The theory describes in a special system language [and in a through way] natural, economic, social structures and their changes. About 20 theoretical directions were highlighted, which reflect various aspects of geographical phenomena investigation of various scales, structuring existing knowledge and allowing to obtain new explanation to [the] facts. A short analysis of the conceptual basis is presented for the directions, used mathematical formalism and tools are introduced, the arrived conclusions are illustrates by examples.

The book could be of interest to the researchers of various scientific fields, which deal with exploration of theoretical knowledge formation foundations and development of a system methodology for concrete problem solving [investigation].

\vspace{1em}

\begin{center}
\textsc{Reviewers of the Russian version}\\{}
Yu.M.~Semenov, Professor of Geography\\{}
G.N.~Konstantinov, Professor of Mathematics\\{}
\vspace{1em}
The Russian version of the book was published with support of\\{} Russian Foundation of Fundamental Research,\\{}
Grant No.~93-05-14076\\{}
``The theory and methods of polysystem analysis''.\\{}
Grant No~96-05-64727\\{}
``Geoinformation processes: research and modeling''.
\end{center}
\vfill{}
\begin{tabular}{lcl}
{}\hspace{0.4\linewidth}{} & & \copyright{} A.K.~Cherkashin, 1997\\
& & \copyright{} Russian Academy of Sciences, 1997\\
& & \copyright{} E.A.~Cherkashin (translator), 2014
\end{tabular}
\end{quote}
\chapter*{Preface}\normalsize
  \addcontentsline{toc}{chapter}{Preface of the editor in chief}
\pagestyle{plain}



A peculiarity of the studies of the holistic picture of the nature had been elaborated as the logical foundation by XIX-th century (A.~Humboldt (?), V.~Dokuchaev and others): a whole visual identity of an uncertain [?unsystem?] form [corresponded] to a [representation] of natural objects as ``many--in--one'' had been used for interpretation the nature patterns [relation, relationships]. As the identities was visual, they were mapped; as they corresponded to concrete phenomena, they were interpreted [via] natural components and causalities. This contributed to sectoral [industrial] branches of geography. Expectations to determination and description of the universal relationship of the phenomena and their object subordination were entrusted to the theoretical basis if the classical landscape science ([beginning--middle] of XX--th century). The notions ontologisation started providing the integrity of perception, resulting in prompt establishment [formation] of the nowadays geography, in particular, the research method for new facts acquisition and scientific conclusions verification. This gave the concrete results characterizing natural and social--economic formations in a new fashion, for example, in the zonal concept. The advance realized the necessary foundation to shift major geographical interpretations from the intuitive ranks to the rank of more precise ones.

In recent years, thanks to a number of objective reasons geography promptly switched from visual observations and generalizations to scientifically organized research in controllable environments. It adopts and develops methodological approaches based on contemporary instruments and new technologies for solving complex geographical problems. Various structures of geographical objects are involved in effects. The essence of the involvements is to be understood. How they should be analyzed? --- As different independent structures followed by their causalities studies, or as whole straight away. Overcoming inertia of intrageographical way of reasoning, a general scientific approach to theory and practice of the complex objects operations have to be developed. A trend is forming that imposes heavy demands to geography metatheory. It recognized disparity [contradictions] epistemological principles and new facts, that almost resulted in discrediting the naive epistemological foundation of classical geography. The known discontentment to its postulates and the necessity perception of their replacement are the outcome of [the trend] as well.

The monograph of Professor A.K.~Cherkashin in this regard has a distinctive character [nature?-]. This is an original effort of a scientist who on the base of the great personal experience of the research work in different knowledge areas seeks to formulate new problems emerging to the geographical science, and to trace ways and methods of their solutions. The center of author's attraction is the theory and the methodology if polysystem analysis and synthesis. Their importance will be understood [by reader] when the objects of the geographical matter [reality] as complex formation, which cannot be described [interpreted] within one theory, has been accepted, with the polysystem analysis being the special field of the [a] constructive research of [the] complex objects. Its methods originated from theoretical and methodological basis of the general theory of systems, the formal and dialectical logics, which are successfully applied in various sciences. The idea of polysystem organization of nature together with social formations created by human being on the background of information lack on their properties, processes and states introduces new elements in the system research of the geographical environment. Firstly, the necessity appears to elaborate [create] a new [diverse] way of systemic qualities classification, methods of their description, recognition and fixation in the science practice. Secondly, and it is essential, in the system research, realizing the objectivity concept in recognition of the systems of various kinds, the informational barrier appears between qualitative and quantitative measurement [commensurability] of ambiguous, unequal and insufficient facts. That's why the informative study of the systems of a certain class must be completed with a set of the research [tools, ware] and methods supplying validity and regularity of the information in a knowledge system. Thirdly, a new approach of creation of an informational basis [base] of the [nature integrity] study and its formalization for description of the elements dependence and the system's outer relationships must be developed.

The ways of the knowledge acquisition and utilization [usage] are developed at the various levels [...of...]. In geography, the studies based on information [collection] techniques and the followed empirical data generalization are developed and still in an active development. Naturally, they will not came up to the total completion as the [any] experience should not [cannot] be interpreted in the abstract[ion] to [a] conceptual and logical basis, which enables the scientific interpretation.

(pg 13 follows)

\pagestyle{headings}
\pagenumbering{arabic}

\include{ch1}
\include{ch2}

\begin{thebibliography}{99}
  \addcontentsline{toc}{chapter}{Bibliography}
\bibitem{lamport} L. Lamport. {\bf \LaTeX \ A Document Preparation System}
Addison-Wesley, California 1986.
\end{thebibliography}
\listoffigures
\listoftables
\include{index}
  \addcontentsline{toc}{chapter}{Index}
\end{document}
